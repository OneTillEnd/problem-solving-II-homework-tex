%%%%%%%%%%%%%%%%%%%%%%%%%%%%%%%%%%%%%%%%%%%%%%%%%%%%%%%%%%%%
% File: hw.tex 						   %
% Description: LaTeX template for homework.                %
%
% Feel free to modify it (mainly the 'preamble' file).     %
% Contact hfwei@nju.edu.cn (Hengfeng Wei) for suggestions. %
%%%%%%%%%%%%%%%%%%%%%%%%%%%%%%%%%%%%%%%%%%%%%%%%%%%%%%%%%%%%

%%%%%%%%%%%%%%%%%%%%%%%%%%%%%%%%%%%%%%%%%%%%%%%%%%%%%%%%%%%%%%%%%%%%%%
% IMPORTANT NOTE: Compile this file using 'XeLaTeX' (not 'PDFLaTeX') %
%
% If you are using TeXLive 2016 on windows,                          %
% you may need to check the following post:                          %
% https://tex.stackexchange.com/q/325278/23098                       %
%%%%%%%%%%%%%%%%%%%%%%%%%%%%%%%%%%%%%%%%%%%%%%%%%%%%%%%%%%%%%%%%%%%%%%

\documentclass[12pt, a4paper, UTF8]{ctexart}
%%%%%%%%%%%%%%%%%%%%%%%%%%%%%%%%%%%
% File: preamble.tex
%%%%%%%%%%%%%%%%%%%%%%%%%%%%%%%%%%%

\usepackage[top = 1.5cm]{geometry}

% Set fonts commands
\newcommand{\song}{\CJKfamily{song}} 
\newcommand{\hei}{\CJKfamily{hei}} 
\newcommand{\kai}{\CJKfamily{kai}} 
\newcommand{\fs}{\CJKfamily{fs}}

\newcommand{\me}[2]{\author{{\bfseries 姓名:}\underline{#1}\hspace{2em}{\bfseries 学号:}\underline{#2}}}

% Always keep this.
\newcommand{\noplagiarism}{
  \begin{center}
    \fbox{\begin{tabular}{@{}c@{}}
      请独立完成作业,不得抄袭。\\
      若得到他人帮助, 请致谢。\\
      若参考了其它资料,请给出引用。\\
      鼓励讨论,但需独立书写解题过程。
    \end{tabular}}
  \end{center}
}

% Each hw consists of three parts:
% (1) this homework
\newcommand{\beginrequired}{\part{作业 (必做部分)}}
\newcommand{\beginoptional}{\part{作业 (选做部分)}}
% (2) corrections (Optional)
\newcommand{\begincorrection}{\part{订正}}
% (3) any feedback (Optional)
\newcommand{\beginfb}{\part{反馈}}

% For math
\usepackage{amsmath}
\usepackage{amsfonts}
\usepackage{amssymb}

% Define theorem-like environments
\usepackage[amsmath, thmmarks]{ntheorem}

\theoremstyle{break}
\theorembodyfont{\song}
\theoremseparator{}
\newtheorem*{problem}{题目}

\theorempreskip{2.0\topsep}
\theoremheaderfont{\kai\bfseries}
\theoremseparator{:}
% \newtheorem*{remark}{注}
\theorempostwork{\bigskip\hrule}
\newtheorem*{solution}{解答}
\theorempostwork{\bigskip\hrule}
\newtheorem*{revision}{订正}

\theoremstyle{plain}
\newtheorem*{cause}{错因分析}
\newtheorem*{remark}{注}

\theoremstyle{break}
\theorempostwork{\bigskip\hrule}
\theoremsymbol{\ensuremath{\Box}}
\newtheorem*{proof}{证明}

\renewcommand\figurename{图}
\renewcommand\tablename{表}

% For figures
% for fig with caption: #1: width/size; #2: fig file; #3: fig caption
\newcommand{\fig}[3]{
  \begin{figure}[htp]
    \centering
      \includegraphics[#1]{#2}
      \caption{#3}
  \end{figure}
}

% for fig without caption: #1: width/size; #2: fig file
\newcommand{\fignocaption}[2]{
  \begin{figure}[htp]
    \centering
    \includegraphics[#1]{#2}
  \end{figure}
}  % modify this file if necessary

%%%%%%%%%%%%%%%%%%%%
\title{第十讲:基本数据结构}
\me{廖玺然}{171860647}
\date{2018 年 5 月 13 日}     % you can specify a date like ``2017年9月30日''.
%%%%%%%%%%%%%%%%%%%%
\begin{document}
\maketitle
%%%%%%%%%%%%%%%%%%%%
\noplagiarism	% always keep this
%%%%%%%%%%%%%%%%%%%%
\beginrequired	% begin ``this homework (hw)'' part

%%%%%%%%%%
\begin{problem}[MA: 2.6]	% NOTE: use '[]' (instead of '()' or '{}') to provide additional information
  Show that the invariance of the ADT stack is not violated by its operations; 
  that is, the three invariances are satisfied before and after each operation.
\end{problem}

% The ``remark'' environments (when needed) must be 
% put before the ``solution''/``revision''/``proof'' environments.
%\begin{remark}	% Optional
%  以下解答参考了书籍/网站/讲义 $\ldots$。

%  \noindent 以下解答是与 XXX 同学讨论得到的。
%\end{remark}

\begin{solution}
  Before PUSH:\\
  \indent Invariance 1: since every time there is only one element pushed onto the stack, 
  it's obvious that $\forall \text{ a, b, t (a, t)}\in \text{elems and (b, t)}\in 
  \text{elems, a}=\text{b}$;\\
  \indent Invariance 2: since the time-stamp is increasing all the time, the current time-stamp 
  is always greater than those recorded in the stack, namely $\forall \text{ (a, t)}\in 
  \text{elems c}>\text{ t}$;\\
  \indent Invariance 3: before PUSH, the stack is certainly valid, so $|\text{elems}|\leq\text{max}$.\\
  After PUSH (suppose we try to push an element p onto the stack at time $\text{t}_{0}$):\\
  \indent Invariance 1: according to invariance 2 before PUSH, $\forall\text{ (a, t) t}\leq\text{t}_{0}$, 
  plus invariance 1 before PUSH, we have $\forall \text{ a, b, t (a, t)}\in \text{elems 
  and (b, t)}\in \text{elems, a}=\text{b}$;\\
  \indent Invariance 2: after pushing p onto the stack, c increases over t$_{0}$, plus 
  invariance 2 before PUSH, we have $\forall \text{ (a, t)}\in \text{elems c}>\text{ t}$;\\
  \indent Invariance 3: if before PUSH, $|\text{elems}|<\text{max}$, after PUSH $|\text{elems}|\leq\text{max}$; 
  if before PUSH, $|\text{elems}|=\text{max}$, after PUSH, $|\text{elems}|=\text{max}$; 
  So we have $|\text{elems}|\leq\text{max}$.
\end{solution}
%%%%%%%%%%

%%%%%%%%%%
\begin{problem}[TC: 10.1-4]
  Rewrite ENQUEUE and DEQUEUE to detect underflow and overflow of a queue.
\end{problem}

\begin{solution}
  ENQUEUE($Q,x$)\\
  \indent if $Q.head == 1$ and $Q.tail == Q.length$\\
  \indent\indent \textbf{error} ``overflow''\\
  \indent if $Q.head == Q.tail + 1$\\
  \indent\indent \textbf{error} ``overflow''\\
  \indent $Q[Q.tail] = x$\\
  \indent if $Q.tail == Q.length$\\
  \indent\indent $Q.tail = 1$\\
  \indent else $Q.tail = Q.tail + 1$\\
  \indent return\\
  DEQUEUE($Q$)\\
  \indent if $Q.head == Q.tail$\\
  \indent\indent \textbf{error} ``underflow''\\
  \indent $x = Q[Q.head]$\\
  \indent if $Q.head == Q.length$\\
  \indent\indent $Q.head = 1$\\
  \indent else $Q.head = Q.head + 1$\\
  \indent return $x$
\end{solution}
%%%%%%%%%%
% \newpage  % continue in a new page
%%%%%%%%%%
\begin{problem}[TC: 10.1-5]
  Whereas a stack allows insertion and deletion of elements at only one end, and 
  a queue allows insertion at one end and deletion at the other end, a \textsl{\textbf{deque}} 
  (double-ended queue) allows insertion and deletion at both ends. Write four 
  $O(1)$-time procedures to insert elements into and delete elements from both 
  ends of a deque implemented by an array.
\end{problem}

% \begin{remark}	
%   Refer to book.
% \end{remark}

\begin{solution}
  (The following array $DQ[1..n]$ can hold at most $n-1$ elements)
  INSERT-AT-RIGHT($DQ,x$)\\
  \indent if $DQ.lend == 1$ and $DQ.rend == DQ.length$\\
  \indent\indent \textbf{error} ``overflow''\\
  \indent if $DQ.lend == DQ.rend + 1$\\
  \indent\indent \textbf{error} ``overflow''\\
  \indent $DQ[DQ.rend] = x$\\
  \indent if $DQ.rend == DQ.length$\\
  \indent\indent $DQ.rend = 1$\\
  \indent else $DQ.rend = DQ.rend + 1$\\
  \indent return\\
  INSERT-AT-LEFT($DQ,x$)\\
  \indent if $DQ.lend == 1$ and $DQ.rend == DQ.length$\\
  \indent\indent \textbf{error} ``overflow''\\
  \indent if $DQ.lend == DQ.rend + 1$\\
  \indent\indent \textbf{error} ``overflow''\\
  \indent if $DQ.lend == 1$\\
  \indent\indent $DQ.lend = DQ.length$\\
  \indent else $DQ.lend = DQ.lend - 1$\\
  \indent $DQ[DQ.lend] = x$\\
  \indent return\\
  DELETE-AT-RIGHT($DQ$)\\
  \indent if $DQ.lend == DQ.rend$\\
  \indent\indent \textbf{error} ``underflow''\\
  \indent if $DQ.rend == 1$\\
  \indent\indent $DQ.rend = DQ.length$\\
  \indent else $DQ.rend = DQ.rend - 1$\\
  \indent $x = DQ[DQ.rend]$\\
  \indent return $x$\\
  DELETE-AT-LEFT($DQ$)\\
  \indent if $DQ.lend == DQ.rend$\\
  \indent\indent \textbf{error} ``underflow''\\
  \indent $x = DQ[DQ.lend]$\\
  \indent if $DQ.lend == DQ.length$\\
  \indent\indent $DQ.lend = 1$\\
  \indent else $DQ.lend = DQ.lend + 1$\\
  \indent return $x$
\end{solution}
%%%%%%%%%%

%%%%%%%%%%
\begin{problem}[TC: 10.1-6]
  Show how to implement a queue using two stacks. Analyze the running time of 
  the queue operations.
\end{problem}

\begin{solution}
  (The two stacks $S1$, $S2$ have the same length $n$)\\
  ENQUEUE($S1,X$)\\
  \indent if $S1.top == S1.length$\\
  \indent\indent \textbf{error} ``overflow''\\
  \indent PUSH($S1,x$)\\
  \indent return\\
  DEQUEUE($S1,S2$)\\
  \indent if STACK-EMPTY($S2$)\\
  \indent\indent if STACK-EMPTY($S1$)\\
  \indent\indent\indent \textbf{error} ``underflow''\\
  \indent\indent do\\
  \indent\indent\indent PUSH($S2,\text{TOP}(S1)$)\\
  \indent\indent until STACK-EMPTY($S1$)\\
  \indent POP($S2$)\\
  The ENQUEUE operation costs $\Theta(1)$ time, the DEQUEUE operation sometimes 
  costs $\Theta(k)$ time where $k$ is the number of elements in stack $S1$, at 
  other time DEQUEUE would also cost $\Theta(1)$ time.
\end{solution}
%%%%%%%%%

%%%%%%%%%
\begin{problem}[TC: 10.2-1]
  Can you implement the dynamic-set operation INSERT on a singly linked list in 
  $O(1)$ time? How about DELETE?
\end{problem}

\begin{solution}
  INSERT is OK. We just need to attach the element to the tail of list.\\
  DELETE is impossible, 'cuz we have to search the list for the element.
\end{solution}
%%%%%%%%%

%%%%%%%%%
\begin{problem}[TC: 10.2-2]
  Implement a stack using a singly linked list $L$. The operations PUSH and POP 
  should still take $O(1)$ time.
\end{problem}

\begin{solution}
  PUSH($L,k$)\\
  \indent node $x =$ \textbf{new} node\\
  \indent $x.key = k$\\
  \indent $x.next = L.head$\\
  \indent $L.head = x$\\
  \indent return\\
  POP($L$)\\
  \indent if $L.head =$ NIL\\
  \indent\indent \textbf{error} ``underflow''\\
  \indent $k = L.head.key$\\
  \indent $L.head = L.head.next$\\
  \indent return $k$
\end{solution}
%%%%%%%%

%%%%%%%%
\begin{problem}[TC: 10.2-3]
  Implement a queue by a singly linked list $L$. The operations ENQUEUE and 
  DEQUEUE should still take $O(1)$ time.
\end{problem}

\begin{solution}
  (We have a node $L.tail$ to track the last element of the list)\\
  INIT-QUEUE($L$)\\
  \indent node $L.head =$ NIL\\
  \indent node $L.tail =$ NIL\\
  ENQUEUE($L,k$)\\
  \indent node $x =$ \textbf{new} node\\
  \indent $x.key = k$\\
  \indent if $L.head ==$ NIL\\
  \indent\indent $L.head = L.tail = x$\\
  \indent\indent return\\
  \indent if $L.head == L.tail$\\
  \indent\indent $L.head.next = x$\\
  \indent\indent $L.tail = x$\\
  \indent\indent return\\
  \indent $L.tail.next = x$\\
  \indent $L.tail = x$\\
  \indent return\\
  DEQUEUE($L$)\\
  \indent if $L.head ==$ NIL\\
  \indent\indent \textbf{error} ``underflow''\\
  \indent $k = L.head.key$\\
  \indent if $L.head == L.tail$\\
  \indent\indent $L.head = L.tail =$ NIL\\
  \indent\indent return $k$\\
  \indent $L.head = L.head.next$\\
  \indent return $k$
\end{solution}
%%%%%%%%%

%%%%%%%%%
\begin{problem}[TC: 10.2-6]
  The dynamic-set operation UNION takes two disjoint sets $S_{1}$ and $S_{2}$ 
  as input, and it returns a set $S = S_{1} \cup S_{2}$ consisting of all the 
  elements of $S_{1}$ and $S_{2}$. The sets $S_{1}$ and $S_{2}$ are usually 
  destroyed by the operation. Show how to support UNION in $O(1)$ time using 
  a suitable list data structure.
\end{problem}

\begin{solution}
  Use two singly linked list. To UNION is just to attach one list's head to 
  the other's tail.
\end{solution}
%%%%%%%

%%%%%%%
\begin{problem}[TC: 10.3-4]
  It is often desirable to keep all elements of a doubly linked list compact in 
  storage, using, for example, the first $m$ index locations in the multiple-array 
  representation. Explain how to implement the procedures ALLOCATE-OBJECT and 
  FREE-OBJECT so that the representation is compact. Assume that there are no 
  pointers to elements of the linked list outside the list itself.
\end{problem}

\begin{solution}
  Keep these rules:
  \begin{itemize}
    \item [1.] $next[i] = i + 1$
    \item [2.] $prev[j] = j - 1$
  \end{itemize}
  To allocate an object, we need to add an element at the tail of array $key$ 
  and add an element at the tail of arrays $prev$ and $next$ respectively 
  according to the rules above;\\
  To deallocate an object, we need to move elements after it in the array $key$ 
  1 position forward and delete the last element of arrays $prev$ and $next$ 
  respectively.
\end{solution}
%%%%%%%

%%%%%%%
\begin{problem}[TC: 10.3-5]
  Let $L$ be a doubly linked list of length $n$ stored in arrays $key$, $prev$ and 
  $next$ of length $m$. Suppose that these arrays are managed by ALLOCATE-OBJECT 
  and FREE-OBJECT procedures that keep a doubly linked free list $F$. Suppose 
  further that of the $m$ items, exactly $n$ are on list $L$ and $m-n$ are on the 
  free list. Write a procedure COMPACTIFY-LIST($L,F$) that, given the list $L$ and 
  the free list $F$, moves the items in $L$ so that they occupy array position 
  $1,2,\ldots,n$ and adjusts the free list $F$ so that it remains correct, occupying 
  array positions $n+1,n+2,\ldots,m$. The running time of your procedure should be 
  $\Theta(n)$, and it should use only a constant amount of extra space. Argue that 
  your procedure is correct.
\end{problem}

\begin{remark}
  以下解答参考了网站:https://ita.skanev.com/10/03/05.html
\end{remark}

\begin{solution}
  (in the following procedure, <variable> == NIL means it has no value\\
  and <variable> = NIL means remove the value once assigned to <variable>)\\
  ($free$ denotes the head index of $F$)\\
  COMPACTIFY-LIST($L,F$)\\
  \indent int $left, right, i$\\
  \indent if $free ==$ NIL\\
  \indent\indent return\\
  \indent $i = free$\\
  \indent while $i\neq$ NIL\\
  \indent\indent $prev[i] = -1$\\
  \indent\indent $i = next[i]$\\
  \indent $left = 1$\\
  \indent $right = m$\\
  \indent while \textbf{true}\\
  \indent\indent while $prev[left]\neq -1$\\
  \indent\indent\indent $left = left + 1$\\
  \indent\indent while $prev[right] == -1$\\
  \indent\indent\indent $right = right - 1$\\
  \indent\indent if $left \geq right$ \textbf{break}\\

  \indent\indent $prev[left] = prev[right]$\\
  \indent\indent $next[left] = next[right]$\\
  \indent\indent $key[left] = key[right]$\\

  \indent\indent $next[right] = left$\\
  \indent\indent $right = right - 1$\\
  \indent\indent $left = left + 1$\\

  \indent $right = right + 1$\\
  \indent for $i$ from 1 to $right - 1$\\
  \indent\indent if $prev[i]\geq right$\\
  \indent\indent\indent $prev[i] = next[prev[i]]$\\
  \indent\indent if $next[i]\geq right$\\
  \indent\indent\indent $next[i] = next[next[i]]$\\
  \indent for $i$ from $right$ to $m - 1$\\
  \indent\indent $next[i] = i + 1$\\
  \indent $next[i] = $ NIL\\
  \indent $free = right$\\
  \indent return
\end{solution}
%%%%%%%%

%%%%%%%%
\begin{problem}[TC: 10.4-2]
  Write an $O(n)$-time recursive procedure that, given an $n$-node binary tree, 
  prints out the key of each node in the tree.
\end{problem}

\begin{solution}
  PRINT-KEY($tree$)\\
  \indent print $tree.key$\\
  \indent if $tree.left\neq$ NIL\\
  \indent\indent PRINT-KEY($tree.left$)\\
  \indent if $tree.right\neq$ NIL\\
  \indent\indent PRINT-KEY($tree.right$)\\
  \indent return
\end{solution}
%%%%%%%

%%%%%%%
\begin{problem}[TC: 10.4-3]
  Write an $O(n)$-time nonrecursive procedure that, given an $n$-node binary tree, 
  prints out the key of each node in the tree. Use a stack as an auxiliary data 
  structure.
\end{problem}

\begin{solution}
  node $stack[n]$\\
  PRINT-KEY($tree$)\\
  \indent $stack[0] = tree$\\
  \indent int $count = 1$\\
  \indent while $count > 0$\\
  \indent\indent $count = count - 1$\\
  \indent\indent $tree = stack[count]$\\
  \indent\indent print $tree.key$\\
  \indent\indent if $tree.left\neq$ NIL\\
  \indent\indent\indent $stack[count] = tree.left$\\
  \indent\indent\indent $count = count + 1$\\
  \indent\indent if $tree.right\neq$ NIL\\
  \indent\indent\indent $stack[count] = tree.right$\\
  \indent\indent\indent $count = count + 1$\\
  \indent return
\end{solution}
%%%%%%%%

%%%%%%%%
\begin{problem}[TC: 10.4-4]
  Write an $O(n)$-time procedure that prints all the keys of an arbitrary rooted 
  tree with $n$-nodes, where the tree is stored using the left-child, right-sibling 
  representation.
\end{problem}

\begin{solution}
  PRINT-KEY($tree$)\\
  \indent print $tree.key$\\
  \indent if $tree.child\neq$ NIL\\
  \indent\indent PRINT-KEY($tree.child$)\\
  \indent if $tree.sibling\neq$ NIL\\
  \indent\indent PRINT-KEY($tree.sibling$)\\
  \indent return
\end{solution}
%%%%%%%%%

%%%%%%%%%
\begin{problem}[TC: 10-3]
\end{problem}

\begin{remark}
  以下解答参考了网站:https://ita.skanev.com/10/problems/03.html
\end{remark}

\begin{solution}
  \begin{itemize}
    \item [a.] The \textbf{for} loop has already taken $t$ times, so the total 
      number of iterations is at least $t$.
    \item [b.] The running time of the \textbf{for} loop is $O(t)$, and the 
      expected running time of the \textbf{while} loop is $O(\text{E}[X_{t}])$, 
      so the expected running time of the procedure is $O(t + \text{E}[X_{t}])$.
    \item [c.] \[ \text{Pr}\{ X_{t}\geq r \} = (\frac{n-r}{n})^{t} = (1 - \frac{r}{n})^{t} \]
      By (C.25) we have
      \[ E(X_{t}) = \sum_{r = 1}^{\infty}\text{Pr}\{ X_{t}\geq r\} = 
      \sum_{r = 1}^{n}\text{Pr}\{X_{t}\geq r \} = \sum_{r = 1}^{n}(1 - \frac{r}{n})^{t} \]
    \item [d.] By (A.11) we have
      \[ \sum_{r = 0}^{n-1}r^{t}\leq \int_{0}^{n} x^{t}dx = \frac{n^{t + 1}}{t + 1} \]
    \item [e.] Through (d) we have
      \begin{align}
        E(X_{t}) & = \sum_{r = 1}^{n}(1 - \frac{r}{n})^{t} \notag \\
                 & = \sum_{r = 0}^{n-1}(\frac{r}{n})^{t} \notag \\
                 & = \frac{1}{n^{t}}\sum_{r = 0}^{n - 1}r^{t} \notag \\
                 & \leq \frac{1}{n^{t}}\cdot\frac{n^{t + 1}}{t + 1} \notag \\
                 & = \frac{n}{t + 1} \notag
      \end{align}
    \item [f.] The expected running time is then $O(t + E(X_{t})) = O(t + \frac{n}{t + 1}) 
      = O(t + \frac{n}{t})$.
    \item [g.] Since COMPACT-LIST-SEARCH minimizes the $O(t + \frac{n}{t})$ running 
      time, we need to find the minimum of $t + \frac{n}{t}$ in the interval $[1,n]$, 
      which is $2\sqrt{n}$, so the expected running time is $O(\sqrt{n})$.
    \item [h.] Because we cannot come to the conclusion in (c) if not all the keys 
      are distinct.
  \end{itemize}

\end{solution}
%%%%%%%%%%%%%%%%%%%%

%%%%%%%%%%%%%%%%%%%%
\end{document}
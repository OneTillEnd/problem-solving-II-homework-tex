%%%%%%%%%%%%%%%%%%%%%%%%%%%%%%%%%%%%%%%%%%%%%%%%%%%%%%%%%%%%
% File: hw.tex 						   %
% Description: LaTeX template for homework.                %
%
% Feel free to modify it (mainly the 'preamble' file).     %
% Contact hfwei@nju.edu.cn (Hengfeng Wei) for suggestions. %
%%%%%%%%%%%%%%%%%%%%%%%%%%%%%%%%%%%%%%%%%%%%%%%%%%%%%%%%%%%%

%%%%%%%%%%%%%%%%%%%%%%%%%%%%%%%%%%%%%%%%%%%%%%%%%%%%%%%%%%%%%%%%%%%%%%
% IMPORTANT NOTE: Compile this file using 'XeLaTeX' (not 'PDFLaTeX') %
%
% If you are using TeXLive 2016 on windows,                          %
% you may need to check the following post:                          %
% https://tex.stackexchange.com/q/325278/23098                       %
%%%%%%%%%%%%%%%%%%%%%%%%%%%%%%%%%%%%%%%%%%%%%%%%%%%%%%%%%%%%%%%%%%%%%%

\documentclass[12pt, a4paper, UTF8]{ctexart}
\input{preamble}  % modify this file if necessary

%%%%%%%%%%%%%%%%%%%%
\title{第九讲:排序与选择}
\me{廖玺然}{171860647}
\date{2018 年 5 月 6 日}     % you can specify a date like ``2017年9月30日''.
%%%%%%%%%%%%%%%%%%%%
\begin{document}
\maketitle
%%%%%%%%%%%%%%%%%%%%
\noplagiarism	% always keep this
%%%%%%%%%%%%%%%%%%%%
\beginrequired	% begin ``this homework (hw)'' part

%%%%%%%%%%
\begin{problem}[TC: 7.1-3]	% NOTE: use '[]' (instead of '()' or '{}') to provide additional information
  Give a brief argument that the running time of \textbf{PARTITION} on 
  a subarray of size $n$ is $\Theta(n)$.
\end{problem}

% The ``remark'' environments (when needed) must be 
% put before the ``solution''/``revision''/``proof'' environments.

\begin{solution}
  Consider the comparison in the loop as the dominant instruction, 
  then we can figure out that the procedure does $n-1$ times of comparison 
  given a subarray of size $n$, so the running time should be $\Theta(n)$.
\end{solution}
%%%%%%%%%%

%%%%%%%%%%
\begin{problem}[TC: 7.2-4]
  Banks often record transactions on an account in order of the times 
  of the transactions, but many people like to receive their bank statements 
  with checks listed in order by check number. People usually write 
  checks in order by check number, and merchants usually cash them with 
  reasonable dispatch. The problem of converting time-of-transaction 
  ordering to check-number ordering is therefore the problem of sorting 
  almost-sorted input. Argue that the procedure \textbf{INSERTION-SORT} 
  would tend to beat the procedure \textbf{QUICKSORT} on this problem.
\end{problem}

\begin{solution}
  Perspective 1: As for people giving out checks, the time ordering is the same as the 
  check-number ordering.\\
  Perspective 2: As for merchants cashing the checks, the check-number ordering is almost 
  the same as the time ordering, and the difference between the two is 
  because those merchants have different speeds in cashing checks. 
  However, the difference in speeds is not so big. So 
  \textbf{every element in the original array is not far from the place where it should be}.\\
  As for the original list and the sorted list, their differences come 
  from Perspective 2.\\
  For such an array of size $n$, the running time of \textbf{INSERTION-SORT} 
  is nearly $\Theta(n)$, the running time of \textbf{QUICKSORT} is nearly 
  $\Theta(n^{2})$, so the former would tend to beat the latter.
\end{solution}
%%%%%%%%%%
% \newpage  % continue in a new page
%%%%%%%%%%
\begin{problem}[TC: 7.3-2]
  When \textbf{RANDOMIZED-QUICKSORT} runs, how many calls are made to 
  the random-number generator \textbf{RANDOM} in the worst case? How 
  about the best case? Give your answer in terms of $\Theta$-notation.
\end{problem}

% \begin{remark}	
%   Refer to book.
% \end{remark}

\begin{solution}
  Denote the total number of calls for a subarray of size $n$ is $f(n)$.\\
  For the worst case, every time the generator choose the largest element 
  or the smallest element in the subarray, namely every time given a 
  subarray of size $n$, the procedure partitions it into $A[r]$ and a 
  subarray of size $n-1$, so\\
  $f(n) = f(n-1) + \Theta(1) = \Theta(n);$\\
  For the best case, every time the generator choose the middle-in-scale 
  element in the subarray, namely every time given a subarray of size $n$, 
  it is partitioned into $A[r]$, a subarray of size $[n/2]$, and another 
  of size $[n/2]-1$, so\\ 
  $f(n) = 2f(n/2) + \Theta(1) = \Theta(n).$
\end{solution}
%%%%%%%%%%

%%%%%%%%%%
\begin{problem}[TC: 7.4-2]
  Show that quicksort's best-case running time is $\Omega(n\lg n)$.
\end{problem}

\begin{solution}
  For the best case, $T(n) = 2T(n) + \Theta(n)$. Using the Master Theorem, 
  we can figure out that $T(n) = \Theta(n\lg n)$, so $T(n) = \Omega(n\lg n)$.
\end{solution}
%%%%%%%%%%%%

%%%%%%%%%%%%
\begin{problem}[TC: 7-5]
  One way to improve the \textbf{RANDOMIZED-QUICKSORT} procedure is to 
  partition around a pivot that is chosen more carefully than by picking 
  a random element from the subarray. One common approach is the \textbf{\textsl{median-of-3}} 
  method: choose the pivot as the median (middle element) of a set of 
  3 elements randomly selected from the subarray. For this problem, let 
  us assume that the elements in the input array $A[1..n]$ are distinct 
  and that $n\geq 3$. We denote the sorted output array by $A'[1..n]$. 
  Using the median-of-3 method to choose the pivot element $x$, define 
  $p_{i} = \text{Pr}\{x = A'[i]\}$.
  \begin{itemize}
    \item[\textbf{\textsl{a. }}] Give an exact formula for $p_{i}$ as a 
      function of $n$ and $i$ for $i = 2,3,\ldots,n-1$.
    \item[\textbf{\textsl{b. }}] By what amount have we increased the 
      likelihood of choosing the pivot as $x = A'[\lfloor(n-1)/2\rfloor]$, 
      the median of $A[1..n]$, compared with the ordinary implementation? 
      Assume that $n\rightarrow\infty$, and give the limiting ratio of 
      these probabilities.
    \item[\textsl{\textbf{c. }}] If we define a ``good'' split to mean 
      choosing the pivot as $x = A'[i]$, where $n/3\leq i\leq 2n/3$, by 
      what amount have we increased the likelihood of getting a good split 
      compared with the ordinary implementation?
    \item[\textsl{\textbf{d. }}] Argue that in the $\Omega(n\lg n)$ running 
      time of quicksort, the median-of-3 method affects only the constant 
      factor.
  \end{itemize}
\end{problem}

\begin{remark}
  以下解答参考了网站:https://ita.skanev.com/07/problems/05.html
\end{remark}

\begin{solution}
  \begin{itemize}
    \item [a.] $p_{i} = \frac{\binom{i-1}{1}\cdot\binom{n-i}{1}}{\binom{n}{3}} 
      = \frac{(i-1)(n-i)}{\binom{n}{3}}$
    \item [b.] $\lim\limits_{n\rightarrow\infty} p_{i}/\frac{1}{n} 
      = \lim\limits_{n\rightarrow\infty} \frac{6(n/2-1)(n-n/2)}{(n-1)(n-2)} 
      = \lim\limits_{n\rightarrow\infty} \frac{6(n^{2} - 2n)}{4(n^{2} - 3n + 2)} 
      = \frac{3}{2},$ so we get $50\%$ improvement.
    \item [c.] After the modification the probability of a good split is
      \begin{align}
        \lim_{n\rightarrow\infty} \sum_{i=n/2}^{i=2n/3} \frac{6(i-1)(n-i)}{n(n-1)(n-2)}
            & = \lim_{n\rightarrow\infty} \frac{6}{n(n-1)(n-2)} \sum_{i=n/3}^{i=2n/3} (i-1)(n-i) \notag \\
            & = \lim_{n\rightarrow\infty} \binom{n}{3} \int_{n/3}^{2n/3} (i-1)(n-i)\text{d}i \notag \\
            & = \lim_{n\rightarrow\infty} \binom{n}{3} \frac{1}{6}\cdot\frac{13}{27}[n^{3} + o(n^{3})] \notag \\
            & = \lim_{n\rightarrow\infty} \frac{13}{27} \frac{n^{3} + o(n^{3})}{n^{3} + o(n^{3})} \notag \\
            & = \frac{13}{27} \notag
      \end{align}
      So we get $\frac{13}{27}\div\frac{1}{3} - 1 = 44\%$ improvement.
    \item [d.] The median-of-3 method does not guarantee to make a good split, so 
      so the $\Omega(n\lg n)$ running time remains.
  \end{itemize}
\end{solution}
%%%%%%%%%%

%%%%%%%%%
\begin{problem}[TC: 8.1-4]
  Suppose that you are given a sequence of $n$ elements to sort. The input 
  sequence consists of $n/k$ subsequence, each containing $k$ elements. 
  The elements in a given subsequence are all smaller than the elements 
  in the succeeding subsequence and larger than the elements in the 
  preceding subsequence. Thus, all that is needed to sort the whole sequence 
  of length $n$ is to sort the $k$ elements in each of the $n/k$ subsequences. 
  Show an $\Omega(n\lg k)$ lower bound on the number of comparisons 
  needed to solve this variant of the sorting problem.
\end{problem}

\begin{solution}
  As each subsequence has $k!$ permutations, there will be totally $(k!)^{n/k}$ 
  permutations. Similar to the proof for Theorem 8.1, we have
  \begin{align}
    2^{h} & \geq (k!)^{n/k} \notag \\
    h     & \geq \lg(k!)^{n/k} \notag \\
          & = (n/k)\lg(k!) \notag \\
          & \geq (n/k)(k/2)\lg(k/2) \notag \\
          & = \frac{1}{2}n\lg k - \frac{1}{2} n \notag \\
          & = \Omega(n\lg k) \notag
  \end{align}
\end{solution}
%%%%%%%%%%%

%%%%%%%%%%%
\begin{problem}[TC: 8.2-4]
  Describe an algorithm that, given $n$ integers in the range 0 to $k$, 
  preprocesses its input and then answer any query about how many of  
  the $n$ integers fall into a range $[a..b]$ in $O(1)$ time. Your algorithm 
  should use $\Theta(n + k)$ preprocessing time.
\end{problem}

\begin{solution}
  Using counting-sort, take $c[i]$ to record the times integers that 
  are not greater than $i$ appears in the original sequence, then we 
  can use $c[b] - c[a-1]$ to denote integers falling into $[a..b]$, 
  specially, $a[-1] = 0$.
\end{solution}
%%%%%%%%%%%

%%%%%%%%%%%
\begin{problem}[TC: 8.3-4]
  Show how to sort $n$ integers in the range 0 to $n^{3} - 1$ in $O(n)$ 
  time.
\end{problem}

\begin{solution}
  Using radix-sort, we take $n$ 3-digit numbers in base $n$, then the 
  running time would be $\Theta(3(n+n)) = \Theta(6n) = O(n)$.
\end{solution}
%%%%%%%%

%%%%%%%%
\begin{problem}[TC: 8.4-2]
  Explain why the worst-case running time for bucket sort is $\Theta(n^{2})$. 
  What simple change to the algorithm preserves its linear average-case 
  running time and makes its worst-case running time $O(n\lg n)$?
\end{problem}

\begin{solution}
  If all the numbers fall into the same bucket and they are in reverse 
  order, then the running time would be the worst-case running time for 
  insertion-sort, which is $O(n^{2})$.\\
  To improve the worst-case efficiency, we may use merge-sort instead 
  of insertion-sort, 'cuz in the worst case, the bucket procedure does 
  not contribute the dominant cost to the running time, so we only have 
  to consider the running time of the sort procedure.
\end{solution}
%%%%%%%%%%%%

%%%%%%%%%%%%
\begin{problem}[TC: 8-2]
  Suppose that we have an array of $n$ data records to sort and that the 
  key to of each record has the value 0 or 1. An algorithm for sorting 
  such a set of records might possess some subset of the following three 
  desirable characteristics:
  \begin{itemize}
    \item [1.] The algorithm runs in $O(n)$ time.
    \item [2.] The algorithm is stable.
    \item [3.] The algorithm sorts in place, using no more than a constant 
      amount of storage space in addition to the original array.
    \item [\textsl{\textbf{a.}}] Give an algorithm that satisfies criteria 
      1 and 2 above.
    \item [\textsl{\textbf{b.}}] Give an algorithm that satisfies criteria 
      1 and 3 above.
    \item [\textsl{\textbf{c.}}] Give an algorithm that satisfies criteria 
      2 and 3 above.
    \item [\textsl{\textbf{d.}}] Can you use any of your sorting algorithms 
      from parts (a)-(c) as the sorting method used in line 2 of \textbf{RADIX-SORT}, 
      so that \textbf{RADIX-SORT} sorts $n$ records with $b$-bit keys in 
      $O(bn)$ time? Explain how or why not.
    \item [\textsl{\textbf{e.}}] Suppose that the $n$ records have keys 
      in the range from 1 to $k$. Show how to modify counting sort so 
      that it sorts the records in place in $O(n+k)$ time. You may use 
      $O(k)$ storage outside the input array. Is your algorithm stable?
  \end{itemize}
\end{problem}

\begin{remark}
  以下解答参考了网站:https://ita.skanev.com/index.html
\end{remark}

\begin{solution}
  \begin{itemize}
    \item [a.] Using counting-sort, we need $\Theta(n)$ space for temporary 
      storage and two variables to respectively count numbers of ones and 
      zeros.
    \item [b.] Using hoare-partition in quicksort which does not separate 
      the pivot element from two subarrays, so that the algorithm will use 
      no more than a constant amount of storage space.
    \item [c.] Bubble-sort is OK.
    \item [d.] Since radix-sort requires stable sort, so (b) is not suitable. 
      (c) takes $\Theta(n^{2})$ time, so it is not suitable either. Only 
      (a) would fit.
    \item [e.]
      $\text{while } i\leq A.length$\\
      $~~~~A[i] \text{ is in the right place}$\\
      $~~~~~~~~i = i + 1$\\
      $~~~~\text{else}$\\
      $~~~~~~~~\text{exchange } A[i] \text{ with the element in the place where }
      A[i]\text{ should be}$
  \end{itemize}
\end{solution}
%%%%%%%

%%%%%%%
\begin{problem}[TC: 9.1-1]
  Show that the second smallest of $n$ elements can be found with $n + \lceil \lg n\rceil - 2$ 
  comparisons in the worst case.
\end{problem}

\begin{solution}
  Algorithm:\\
  We divide the elements into pairs and do comparisons within each group. Repeat 
  this process, we can find the smallest element with $n - 1$ comparisons.\\
  We can easily figure out that the second smallest element is among the $\lceil\lg n\rceil$ 
  elements that have been compared with the smallest element, so we simply do 
  $\lceil\lg n\rceil - 1$ comparisons among them to find the second smallest element.\\
  So in total, we need $n + \lceil\lg n\rceil - 2$ comparisons in the worst case.
\end{solution}
%%%%%%%%

%%%%%%%%
\begin{problem}[TC: 9.3-7]
  Describe an $O(n)$-time algorithm that, given a set $S$ of $n$ distinct 
  numbers and a positive integer $k\leq n$, determines the $k$ numbers 
  in $S$ that are closest to the median of $S$.
\end{problem}

\begin{solution}
  Algorithm:\\
  1. Find the median of the set in $O(n)$ time;\\
  2. Figure out the distance of every element to the median in $O(n)$ time;\\
  3. Find the $k$-th closest element to the median in $O(n)$ time;\\
  4. Select elements closer than or as close as the element in (3) to the median 
  in $O(n)$ time.\\
  Then the total running time would be $4O(n) = O(n)$.
\end{solution}
%%%%%%%%%%%%%%%%%%%%
\end{document}
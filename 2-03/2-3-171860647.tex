%%%%%%%%%%%%%%%%%%%%%%%%%%%%%%%%%%%%%%%%%%%%%%%%%%%%%%%%%%%%
% File: hw.tex 						   %
% Description: LaTeX template for homework.                %
%
% Feel free to modify it (mainly the 'preamble' file).     %
% Contact hfwei@nju.edu.cn (Hengfeng Wei) for suggestions. %
%%%%%%%%%%%%%%%%%%%%%%%%%%%%%%%%%%%%%%%%%%%%%%%%%%%%%%%%%%%%

%%%%%%%%%%%%%%%%%%%%%%%%%%%%%%%%%%%%%%%%%%%%%%%%%%%%%%%%%%%%%%%%%%%%%%
% IMPORTANT NOTE: Compile this file using 'XeLaTeX' (not 'PDFLaTeX') %
%
% If you are using TeXLive 2016 on windows,                          %
% you may need to check the following post:                          %
% https://tex.stackexchange.com/q/325278/23098                       %
%%%%%%%%%%%%%%%%%%%%%%%%%%%%%%%%%%%%%%%%%%%%%%%%%%%%%%%%%%%%%%%%%%%%%%

\documentclass[11pt, a4paper, UTF8]{ctexart}
\input{preamble}  % modify this file if necessary

%%%%%%%%%%%%%%%%%%%%
\title{第三讲:组合与计数}
\me{廖玺然}{171860647}{计算机科学与技术系}
\date{2018 年 3 月18 日}     % you can specify a date like ``2017年9月30日''.
%%%%%%%%%%%%%%%%%%%%
\begin{document}
\maketitle
%%%%%%%%%%%%%%%%%%%%
\noplagiarism	% always keep this
%%%%%%%%%%%%%%%%%%%%
\beginthishw	% begin ``this homework (hw)'' part

%%%%%%%%%%
\begin{problem}[CS: 1.1.13]	% NOTE: use '[]' (instead of '()' or '{}') to provide additional information
  Suppose on Day 1 you receive one penny, and, for $i > 1$, on Day $i$ you receive 
  twice as many pennies as you did on Day $i - 1$. How many pennies will you have on 
  Day 20? How many will you have on Day $n$? Can you justify your answer by using the 
  sum or product principle?
\end{problem}

% The ``remark'' environments (when needed) must be 
% put before the ``solution''/``revision''/``proof'' environments.
%\begin{remark}	% Optional
%  以下解答参考了书籍/网站/讲义 $\ldots$。

%  \noindent 以下解答是与 XXX 同学讨论得到的。
%\end{remark}

\begin{solution}
  On Day $20$, there will be $2^{20} - 1$ pennies.\\
  On Day $n$, there will be $2^{n} - 1$ pennies.\\
  Consider the pennies received on Day $i$ as set $P_{i}$, the total in Day $n$ 
  as set $S_{n}$, according to the sum principle, $|S_{n}| = \sum_{i = 1}^{n} |P_{i}|$, 
  since $|P_{i}| = 2^{i - 1}$, we have $|S_{n}| = 2^{n} - 1$, which completes the proof.
\end{solution}
%%%%%%%%%%

%%%%%%%%%%
\begin{problem}[CS: 1.2.5]
  Assuming $k \leq n$, in how many ways can we pass out $k$ distinct pieces of fruit 
  to $n$ children if each child may get at most one piece? What if $k > n$? Assume 
  for both questions that we pass out all the fruit.
\end{problem}

\begin{solution}
  1) $k \leq n$: the process is like picking out all the $k$-element subsets of an 
  $n$-element set, then figuring out all the permutations of each subset 
  respectively, thus we have $\binom{n}{k} \cdot k! = \frac{n!}{(n-k)!}$ ways.\\
  2) $k \geq n$: the process is like figuring out all the $n$-element permutations of 
  a $k$-element set, thus we have $k(k-1)\cdots(k-n+1) = \frac{k!}{(k-n)!}$ ways.
\end{solution}
%%%%%%%%%%
% \newpage  % continue in a new page
%%%%%%%%%%
\begin{problem}[CS: 1.2.15]
  A tennis club has $2n$ members. We want to pair up the members by twos for singles 
  matches. In how many ways can we pair up all the members of the club? Suppose that 
  in addition to specifying who plays whom, we also determine who serves first for 
  each pairing. Now in how many ways can we specify our pairs?
\end{problem}

% \begin{remark}	
%   Refer to book.
% \end{remark}

\begin{solution}
  1) First we pick out all the permutations of a $2n$-element set, namely $(2n)!$,\\
  Considering that this process includes those permutations that have the same pair
  (though in different order), we have $\frac{(2n)!}{2^{n}}$ permutations excluding 
  the repeated ones,\\
  Finally we pair up the adjacent two from head to tail, and exclude those sequence 
  with same pairs in different order, then we can get $\frac{(2n)!}{n! \cdot 2^{n}}$ 
  choices.\\
  2) For one way of division, we have $2^{n}$ ways to specify, so in total we have 
  $\frac{(2n)!}{n!}$ ways.
\end{solution}
%%%%%%%%%%

%%%%%%%%%%
\begin{problem}[CS: 1.5.4]
  Use multisets to determine the number of ways to pass out $k$ identical apples to 
  $n$ children. Assume that a child may get more than one apple.
\end{problem}

\begin{solution}
  We may suppose that the apples have been numbered, then according to the bookcase 
  example, we have $\frac{(n+k-1)!}{(n-1)!}$ ways of distribution, but in fact the 
  apples are identical, so the result shoud be $\frac{(n+k-1)!}{k! \cdot (n-1)!}$.
\end{solution}
%%%%%%%%%%%%

%%%%%%%%%%%
\begin{problem}[CS: 1.5.12]
  A standard notation for the number of partitions of an $n$-element set into $k$ 
  classes is $S(n,k)$. Because the empty family of subsets of the empty set is a 
  partition of the empty set, $S(0,0)$ is 1. In addition, $S(n,0)$ is 0 for $n > 0$, 
  because there are no partitions of a nonempty set into no parts. $S(1,1)$ is 1.\\
  a. Explain why $S(n,n)$ is 1 for all $n > 0$. Explain why $S(n,1)$ is 1 for all 
  $n > 0$.\\
  b. Explain why $S(n,k) = S(n-1,k-1) + kS(n-1,k)$ for $1 < k < n$.\\
  c. Make a table like Table 1.1 that shows the values of $S(n,k)$ for values of $n$ 
  and $k$ ranging from 1 to 6.
\end{problem}

\begin{solution}
  a. $S(n,n)$ indicates the number of ways to pick out $n$-element subsets of an 
  $n$-element set, and there's only one such subset, namely itself. \\
  $~~~~S(n,1)$ indicates the number of ways to pick out $1$-element subsets of an 
  $n$-element set, that is picking out 1 element at a time, so there's only 1 way.\\
  b. $S(n,k)$ could be generated from two aspects: add 1 class beside $k-1$ classes 
  partitioning an $n-1$-element set, which has only 1 way, or add 1 element to one 
  of the $k$ classes partitioning an $n-1$-element set, which has $k$ ways. So we 
  can conclude that $S(n,k) = S(n-1,k-1) + kS(n-1,k)$.\\
  c. \fignocaption{width = 1.0\textwidth}{figs/table_1.png}
\end{solution}
%%%%%%%%%

%%%%%%%%%%
\begin{problem}[CS: 1.5.14]
  Consider the following C++ function to compute $\binom{n}{k}$.\\
  \indent int pascal(int n, int k)\\
  \indent \{\\
  \indent \indent if (n < k)\\
  \indent \indent \{\\
  \indent \indent \indent cout << "error: n < k" << endl;\\
  \indent \indent \indent exit(1);\\
  \indent \indent \}\\
  \indent \indent if ((k==0)||(n==k)\\
  \indent \indent \indent return 1;\\
  \indent \indent return pascal(n-1,k-1) + pascal(n-1,k);\\
  \indent \}\\
  Enter this code, compile it, and run it (you will need to create a simple main 
  program that calls it). Run it on larger and larger values of $n$ and $k$, and 
  observe the running time of the program. It should be surprisingly slow. Why is 
  it so slow? Can you write a different function to compute $\binom{n}{k}$ that is 
  \textsl{significantly} faster? Why is your new version faster?
\end{problem}

\begin{solution}
  As the times of recursion increase, the two addends may become incredibly large, 
  so the process would be really slow.\\
  modified version:\\
  \indent int pascal(int n, int k)\\
  \indent \{\\
  \indent \indent int coefficient = 1;\\
  \indent \indent for(int i = 1, i <= k, ++i)\\
  \indent \indent \{\\
  \indent \indent \indent coefficient *= ((n-k+i)/i);\\
  \indent \indent \}\\
  \indent \indent return coefficient;\\
  \indent \}\\
  This version is a lot faster cuz each factor is not that large.
\end{solution}
%%%%%%%%%%%%%%%%%%%%
%\begincorrection	% begin the ``correction'' part (Optional)

%%%%%%%%%%
%\begin{problem}[题号]
%  题目。
%\end{problem}

%\begin{cause}
%  简述错误原因(可选)。
%\end{cause}

% Or use the ``solution'' environment
%\begin{revision}
%  正确解答。
%\end{revision}
%%%%%%%%%%
%%%%%%%%%%%%%%%%%%%%
%\beginfb	% begin the feedback section (Optional)

%你可以写:
%\begin{itemize}
%  \item 对课程及教师的建议与意见
%  \item 教材中不理解的内容
%  \item 希望深入了解的内容
%  \item 等
%\end{itemize}
%%%%%%%%%%%%%%%%%%%%
\end{document}
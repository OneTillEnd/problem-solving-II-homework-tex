%%%%%%%%%%%%%%%%%%%%%%%%%%%%%%%%%%%%%%%%%%%%%%%%%%%%%%%%%%%%
% File: hw.tex 						   %
% Description: LaTeX template for homework.                %
%
% Feel free to modify it (mainly the 'preamble' file).     %
% Contact hfwei@nju.edu.cn (Hengfeng Wei) for suggestions. %
%%%%%%%%%%%%%%%%%%%%%%%%%%%%%%%%%%%%%%%%%%%%%%%%%%%%%%%%%%%%

%%%%%%%%%%%%%%%%%%%%%%%%%%%%%%%%%%%%%%%%%%%%%%%%%%%%%%%%%%%%%%%%%%%%%%
% IMPORTANT NOTE: Compile this file using 'XeLaTeX' (not 'PDFLaTeX') %
%
% If you are using TeXLive 2016 on windows,                          %
% you may need to check the following post:                          %
% https://tex.stackexchange.com/q/325278/23098                       %
%%%%%%%%%%%%%%%%%%%%%%%%%%%%%%%%%%%%%%%%%%%%%%%%%%%%%%%%%%%%%%%%%%%%%%

\documentclass[12pt, a4paper, UTF8]{ctexart}
%%%%%%%%%%%%%%%%%%%%%%%%%%%%%%%%%%%
% File: preamble.tex
%%%%%%%%%%%%%%%%%%%%%%%%%%%%%%%%%%%

\usepackage[top = 1.5cm]{geometry}

% Set fonts commands
\newcommand{\song}{\CJKfamily{song}} 
\newcommand{\hei}{\CJKfamily{hei}} 
\newcommand{\kai}{\CJKfamily{kai}} 
\newcommand{\fs}{\CJKfamily{fs}}

\newcommand{\me}[2]{\author{{\bfseries 姓名:}\underline{#1}\hspace{2em}{\bfseries 学号:}\underline{#2}}}

% Always keep this.
\newcommand{\noplagiarism}{
  \begin{center}
    \fbox{\begin{tabular}{@{}c@{}}
      请独立完成作业,不得抄袭。\\
      若得到他人帮助, 请致谢。\\
      若参考了其它资料,请给出引用。\\
      鼓励讨论,但需独立书写解题过程。
    \end{tabular}}
  \end{center}
}

% Each hw consists of three parts:
% (1) this homework
\newcommand{\beginrequired}{\part{作业 (必做部分)}}
\newcommand{\beginoptional}{\part{作业 (选做部分)}}
% (2) corrections (Optional)
\newcommand{\begincorrection}{\part{订正}}
% (3) any feedback (Optional)
\newcommand{\beginfb}{\part{反馈}}

% For math
\usepackage{amsmath}
\usepackage{amsfonts}
\usepackage{amssymb}

% Define theorem-like environments
\usepackage[amsmath, thmmarks]{ntheorem}

\theoremstyle{break}
\theorembodyfont{\song}
\theoremseparator{}
\newtheorem*{problem}{题目}

\theorempreskip{2.0\topsep}
\theoremheaderfont{\kai\bfseries}
\theoremseparator{:}
% \newtheorem*{remark}{注}
\theorempostwork{\bigskip\hrule}
\newtheorem*{solution}{解答}
\theorempostwork{\bigskip\hrule}
\newtheorem*{revision}{订正}

\theoremstyle{plain}
\newtheorem*{cause}{错因分析}
\newtheorem*{remark}{注}

\theoremstyle{break}
\theorempostwork{\bigskip\hrule}
\theoremsymbol{\ensuremath{\Box}}
\newtheorem*{proof}{证明}

\renewcommand\figurename{图}
\renewcommand\tablename{表}

% For figures
% for fig with caption: #1: width/size; #2: fig file; #3: fig caption
\newcommand{\fig}[3]{
  \begin{figure}[htp]
    \centering
      \includegraphics[#1]{#2}
      \caption{#3}
  \end{figure}
}

% for fig without caption: #1: width/size; #2: fig file
\newcommand{\fignocaption}[2]{
  \begin{figure}[htp]
    \centering
    \includegraphics[#1]{#2}
  \end{figure}
}  % modify this file if necessary

%%%%%%%%%%%%%%%%%%%%
\title{第七讲:离散概率基础}
\me{廖玺然}{171860647}
\date{2018 年 4 月 19 日}     % you can specify a date like ``2017年9月30日''.
%%%%%%%%%%%%%%%%%%%%
\begin{document}
\maketitle
%%%%%%%%%%%%%%%%%%%%
\noplagiarism	% always keep this
%%%%%%%%%%%%%%%%%%%%
\beginrequired	% begin ``this homework (hw)'' part

%%%%%%%%%%
\begin{problem}[CS: 5.1.10]	% NOTE: use '[]' (instead of '()' or '{}') to provide additional information
  How many five-card hands chosen from a standard deck of playing cards consist 
  of five cards in a row (such as the nine of diamonds, ten of clubs, jack of 
  clubs, queen of hearts, and king of spades)? Such a hand is called a straight. 
  What is the probability that a five-card hand is a straight? Explore whether 
  you get the same answer by using five-element sets as your model of hands or 
  five-element permutations as your model of hands.
\end{problem}

% The ``remark'' environments (when needed) must be 
% put before the ``solution''/``revision''/``proof'' environments.
%\begin{remark}	% Optional
%  以下解答参考了书籍/网站/讲义 $\ldots$。

%  \noindent 以下解答是与 XXX 同学讨论得到的。
%\end{remark}

\begin{solution}
  There are $10 \times 4^{5} = 10240$ straights.\\
  By opinion of sets:\\
  $P_{1} = \frac{10 \times 4^{5}}{\binom{52}{5}} = \frac{128}{32487}.$\\
  By opinion of permutations:\\
  $P_{2} = \frac{10 \times 4^{5} \times A_{5}^{5}}{\binom{52}{5} \times A_{5}^{5}} 
  = \frac{128}{32487}.$\\
  they are the same.
\end{solution}
%%%%%%%%%%

%%%%%%%%%%
\begin{problem}[CS: 5.1.12]
  A die is made of a cube with a square painted on one side, a circle painted 
  on two sides, and a triangle on three sides. If the die is rolled twice, what 
  is the probability that the two shapes you see on top are the same?
\end{problem}

\begin{solution}
  $P = \frac{1^{2} + 2^{2} + 3^{2}}{6^{2}} = \frac{7}{18}.$
\end{solution}
%%%%%%%%%%
% \newpage  % continue in a new page
%%%%%%%%%%
\begin{problem}[CS: 5.2.4]
  A bowl contains two red, two white, and two blue balls. If you remove two 
  balls, what is the probability that at least one is red or white? Compute 
  the probability that at least one is red.
\end{problem}

% \begin{remark}	
%   Refer to book.
% \end{remark}

\begin{solution}
  $P_{1} = \frac{\binom{4}{1}\cdot\binom{2}{1} + \binom{4}{2}}{\binom{6}{2}} 
  = \frac{14}{15};$\\
  $P_{2} = \frac{\binom{2}{1}\cdot\binom{4}{1} + \binom{2}{2}}{\binom{6}{2}} 
  = \frac{3}{5}.$
\end{solution}
%%%%%%%%%%

%%%%%%%%%%
\begin{problem}[CS: 5.2.10]
  If you are hashing $n$ keys into a hash table with $k$ locations, what is 
  the probability that every location gets at least one key?
\end{problem}

\begin{solution}
  If $n < k$, it's impossible;\\
  If $n \geq k$, the number of hashings equals the number of functions from 
  an $n$-element set onto a $k$-element set, according to Theorem 5.4, that is
  \[ \sum_{i=0}^{k} (-1)^{i} \binom{k}{i} (k - i)^{n}. \]
\end{solution}
%%%%%%%%%%

%%%%%%%%%%
\begin{problem}[CS: 5.3.2]
  In three flips of a coin, is the event that two flips in a row are heads 
  independent of the event that there is an even number of heads?
\end{problem}

\begin{solution}
  Denote `two flips in a row are heads" as event $E$, `there is an even number 
  of heads' as event $F$, then\\
  $P(E \cap F) = (\frac{1}{2})^{2} \times \frac{1}{2} + \frac{1}{2} \times (\frac{1}{2})^{2} 
  = \frac{1}{4};$\\
  $P(E) = (\frac{1}{2})^{2} + (\frac{1}{2})^{2} - (\frac{1}{2})^{3} = \frac{3}{8};$\\
  $P(F) = \binom{3}{1} \cdot (\frac{1}{2})^{3} + (\frac{1}{2})^{3} = \frac{1}{4};$\\
  $P(E \cap F) \neq P(E)P(F)$, so the two events are not independent of each 
  other.
\end{solution}
%%%%%%%%%%

%%%%%%%%%%
\begin{problem}[CS: 5.3.6]
  Assume that on a true-false test, students will answer correctly any question 
  on a subject that they know. Assume students guess at answers they do not know. 
  For students who know $60\%$ of the material in a course, what is the 
  probability that they will answer a question correctly? What is the probability 
  that they will know the answer to a question they answer correctly?
\end{problem}

\begin{solution}
  Denote `they answer a question correctly' as event $F$, `they know the answer' 
  as event $E$, then\\
  $P(F) = 60\% + 50\% \times 40\% = 80\%;$\\
  $P(E \cap F) = 60\%;$\\
  $P(E | F) = \frac{P(E \cap F)}{P(F)} = \frac{3}{4}.$
\end{solution}
%%%%%%%%%%%%

%%%%%%%%%%%%
\begin{problem}[CS: 5.3.12]
  In a family consisting of a mother, father and two children of different ages, 
  what is the probability that the family has two girls, given that one of the 
  children is a girl? What is the probability that the children are both boys, 
  given that the older child is a boy?
\end{problem}

\begin{solution}
  A child's gender has nothing to do with others, so whether a child is a boy 
  or a girl is independent of the other, then\\
  $P_{1} = \frac{1}{2};$\\
  $P_{2} = \frac{1}{2}.$
\end{solution}
%%%%%%%%

%%%%%%%%
\begin{problem}[CS: 5.4.4]
  Assuming that the process of answering the questions on a five-question quiz 
  is an independent trials process and that a student has a probability $.8$ 
  of answering any given question correctly, what is the probability of one 
  particular sequence of four correct answers and one incorrect answer? What 
  is the probability that a student answers exactly four questions correctly?
\end{problem}

\begin{solution}
  $P_{1} = .8^{4} \times (1 - .8) = \frac{256}{3125};$\\
  $P_{2} = \binom{5}{4} \times .8^{4} \times (1 - .8) = \frac{256}{625}.$
\end{solution}
%%%%%%%%%

%%%%%%%%%
\begin{problem}[CS: 5.4.10]
  What is the expected value of the constant random variable $X$ that has 
  $X(s) = c$ for every number $s$ of the sample space?
\end{problem}

\begin{solution}
  \[E(X) = \sum_{s} c\cdot P(s) = c\cdot \sum_{s} P(s) = c \cdot 1 = c.\]
\end{solution}
%%%%%%%%%%

%%%%%%%%%%
\begin{problem}[CS: 5.4.12]
  Solve Problem 11 for the case of a student taking a multiple-choice test 
  with five choices for each answer and randomly guessing when she doesn't 
  know the answer.
\end{problem}

\begin{solution}
  Suppose the percentage of known material is $k$, then the percentage of correct 
  answers is $k + \frac{1-k}{5} = \frac{1+4k}{5}$, the percentage of incorrect 
  answers is $1 - \frac{1+4k}{5} = \frac{4-4k}{5}$.\\
  Let $X(y) = \frac{1+4k}{5} - y\cdot \frac{4-4k}{5}$, according to the requirement, 
  we hope that $E(X) = k$, namely
  \[\sum_{y} P(y) \cdot (\frac{1+4k}{5} - y\cdot \frac{4-4k}{5}) = k,\]
  We can easily find a constant $y = \frac{1}{4}$ that satisfies the equation.\\
  So we should let $y = \frac{1}{4}$.
\end{solution}
%%%%%%%%%

%%%%%%%%%
\begin{problem}[CS: 5.4.15]
  Prove Theorem 5.11.
\end{problem}

\begin{solution}
  \begin{align}
    E(cX) & = \sum_{s:s\in S} c\cdot X(s) P(s) \notag \\
          & = c\cdot \sum_{s:s\in S} X(s)P(s) \notag \\
          & = cE(X) \notag
  \end{align}
\end{solution}
%%%%%%%%%%%%%%%%%%%
\end{document}
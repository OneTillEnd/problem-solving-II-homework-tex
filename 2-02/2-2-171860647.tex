%%%%%%%%%%%%%%%%%%%%%%%%%%%%%%%%%%%%%%%%%%%%%%%%%%%%%%%%%%%%
% File: hw.tex 						   %
% Description: LaTeX template for homework.                %
%
% Feel free to modify it (mainly the 'preamble' file).     %
% Contact hfwei@nju.edu.cn (Hengfeng Wei) for suggestions. %
%%%%%%%%%%%%%%%%%%%%%%%%%%%%%%%%%%%%%%%%%%%%%%%%%%%%%%%%%%%%

%%%%%%%%%%%%%%%%%%%%%%%%%%%%%%%%%%%%%%%%%%%%%%%%%%%%%%%%%%%%%%%%%%%%%%
% IMPORTANT NOTE: Compile this file using 'XeLaTeX' (not 'PDFLaTeX') %
%
% If you are using TeXLive 2016 on windows,                          %
% you may need to check the following post:                          %
% https://tex.stackexchange.com/q/325278/23098                       %
%%%%%%%%%%%%%%%%%%%%%%%%%%%%%%%%%%%%%%%%%%%%%%%%%%%%%%%%%%%%%%%%%%%%%%

\documentclass[11pt, a4paper, UTF8]{ctexart}
\input{preamble}  % modify this file if necessary

%%%%%%%%%%%%%%%%%%%%
\title{第二讲:算法的效率}
\me{廖玺然}{171860647}{计算机科学与技术系}
\date{\today}     % you can specify a date like ``2017年9月30日''.
%%%%%%%%%%%%%%%%%%%%
\begin{document}
\maketitle
%%%%%%%%%%%%%%%%%%%%
\noplagiarism	% always keep this
%%%%%%%%%%%%%%%%%%%%
\beginthishw	% begin ``this homework (hw)'' part

%%%%%%%%%%
\begin{problem}[DH: 6.1]	% NOTE: use '[]' (instead of '()' or '{}') to provide additional information
  Consider the following salary computation problem. The input consists of a number \textsl{N}, 
  a list, \textsl{BT}(1),..., \textsl{BT}(\textsl{N}), of before-tax salaries, and two 
  tax rates, \textsl{Rh} for salaries that are larger than \textsl{M}, and \textsl{Rl} 
  for salaries that are no larger than \textsl{M}. Here \textsl{Rh} and \textsl{Rl} are 
  positive fractions, i.e., 0 < \textsl{Rh}, \textsl{Rl} < 1. It is required to compute 
  a list, \textsl{AT}(1),..., \textsl{AT}(\textsl{N}), of after-tax salaries, and two 
  sums, \textsl{Th} and \textsl{Tl}, containing the total taxes paid according to the 
  two rates, respectively. Here is a solution to the problem. It calculates the after-tax 
  salaries first and then the tax totals:\\
  $~~~~\text{for } I \text{ from 1 to } N \text{ do the following:}$\\
  $~~~~~~~~\text{if } BT(I) > M \text{ then } AT(I) \gets BT(I) \times (1 - Rh);$\\
  $~~~~~~~~\text{otherwise } AT(I) \gets BT(I) \times (1 - Rl);$\\
  $~~~~Th \gets 0;$\\
  $~~~~Tl \gets 0;$\\
  $~~~~\text{for } I \text{ from 1 to } N \text{ do the following:}$\\
  $~~~~~~~~\text{if } BT(I) > M \text{ then } Th \gets Th + BT(I) \times Rh;$\\
  $~~~~~~~~\text{otherwise } Tl \gets Tl + BT(I) \times Rl.$\\
  (a) Suggest a series of transformations that will make the program as efficient as 
  possible, by minimizing both the number of arithmetical operations and the number of 
  comparisons. Estimate the improvement to these complexity measures that is gained by 
  each of your transformations.\\
  (b) How would you further improve the program, if you are guaranteed that no before-tax 
  salary is strictly less than \textsl{M}(i.e., there might be before-tax salaries of 
  exactly \textsl{M}, but not less)? How would you characterize the rate of improvement 
  in this case?
\end{problem}

% The ``remark'' environments (when needed) must be 
% put before the ``solution''/``revision''/``proof'' environments.
%\begin{remark}	% Optional
%  以下解答参考了书籍/网站/讲义 $\ldots$。

%  \noindent 以下解答是与 XXX 同学讨论得到的。
%\end{remark}

\begin{solution}
  (a) Improved version:\\
  $~~~~RH \gets 1 - Rh;~RL \gets 1 - Rl;$\\
  $~~~~Th \gets 0;~Tl \gets 0;$\\
  $~~~~\text{for } I \text{ from 1 to } N \text{ do the following:}$\\
  $~~~~~~~~\text{if } BT(I) > M \text{ then}$\\
  $~~~~~~~~~~~~AT(I) \gets BT(I) \times RH;$\\
  $~~~~~~~~~~~~Th \gets Th + BT(I) \times Rh;$\\
  $~~~~~~~~\text{otherwise}$\\
  $~~~~~~~~~~~~AT(I) \gets BT(I) \times RL;$\\
  $~~~~~~~~~~~~Tl \gets Tl + BT(I) \times Rl.$\\
  不考虑其他改动的情况下,\\
  1) 减少循环数目降低了$N$的时间消耗;\\
  2) 先计算税后税前工资之比减少了$N - 1$次操作数。\\
  (b) $RH \gets 1 - Rh;$\\
  $~~~~~~Sl \gets M \times (1 - Rl);~TL \gets M \times Rl$\\
  $~~~~~~Th \gets 0;~Tl \gets 0;$\\
  $~~~~~~\text{for } I \text{ from 1 to } N \text{ do the following:}$\\
  $~~~~~~~~~~\text{if } BT(I) > M \text{ then}$\\
  $~~~~~~~~~~~~~~AT(I) \gets BT(I) \times RH;$\\
  $~~~~~~~~~~~~~~Th \gets Th + BT(I) \times Rh;$\\
  $~~~~~~~~~~\text{otherwise}$\\
  $~~~~~~~~~~~~~~AT(I) \gets Sl;$\\
  $~~~~~~~~~~~~~~Tl \gets Tl + TL.$\\
  先计算了工资为$M$元的人税后所得和应缴税款,减少了$2(N - 1)$次操作数,时间消耗减少了
  约33\%(仅考虑计算操作).
\end{solution}
%%%%%%%%%%

%%%%%%%%%%
\begin{problem}[DH: 6.8]
  Prove that the following algorithms are of linear-time complexity:\\
  (a) The algorithm for counting sentences containing the word "money", presented in 
  Chapter 2.\\
  (b) The algorithm for maximal polygonal distance, presented in Chapter 4.
\end{problem}

\begin{solution}
  (a) 这个搜索算法总会遍历文段中所有单词,其他操作数与“money”的出现次数相当,与单词总数
  相比可以忽略,设单词总数为$N$,则时间复杂度为$O(N)$,即线性时间复杂度。\\
  (b) 这个算法的循环次数与多边形的边数相等,除第一次循环只需执行1次比较外,其余循环内均需
  进行2次比较,设多边形边数为$N$,操作数约为$3N$,因此时间复杂度为$O(N)$,即线性时间复杂度。
\end{solution}
%%%%%%%%%%
% \newpage  % continue in a new page
%%%%%%%%%%
\begin{problem}[UD: 6.10]
  Analyze the worst-case time complexity of the algorithms for traversing and processing 
  information stored in trees that you were asked to design in Exercises 4.3 and 4.2.
\end{problem}

% \begin{remark}	
%   Refer to book.
% \end{remark}

\begin{solution}
  算法未完成,下次作业订正麻烦老师了,致歉。
\end{solution}
%%%%%%%%%%

%%%%%%%%%%%%%%
\begin{problem}[DH: 6.13]
  Prove a lower bound of $O(N \times \text{log}_{2}N)$ on the time complexity of any 
  comparison-based sorting algorithm.
\end{problem}

\begin{remark}
  以下解答参考了GeeksforGeeks: Lower bound for comparison based sorting algorithm\\
  https://www.geeksforgeeks.org/lower-bound-on-comparison-based-sorting-algorithms/
\end{remark}

\begin{solution}
  我们可以把这种比较排序算法画成一棵树,叶节点是$N$个元素的所有排列,叶节点以上的节点表示
  对两个元素顺序的判断(因此有两个分支),因此我们可得以下两个事实:\\
  1) 叶节点至少有$N!$个;\\
  2) 假设执行算法需要对最大比较操作数为$x$,则树的深度最大为$x$,而深度为$x$的树最多有
  $2^{x}$个叶节点;\\
  因此有$n! \leq 2^{x}$,\\
  两边取以2为底的对数得$\text{log}_{2}(n!) \leq x$,\\
  由于$\text{log}_{2}(n!) = \Theta(n\text{log}n)$,\\
  所以$x = \Omega(n\text{log}_{2}n)$,\\
  得证。
\end{solution}
%%%%%%%%%%%%%

%%%%%%%%%%%%
\begin{problem}[DH: 6.18]
  Design an algorithm \textsl{LG}1, with input integers $m, n > 1$, that calculates 
  $\text{lg}_{m}n$ by repeatedly calculating the powers $m^{0}, m^{1},\cdots,m^{k}$, 
  until a number $k$ is found satisfying $m^{k} \leq n \leq m^{k+1}$. Analyze the time 
  and space complexity of \textsl{LG}1.
\end{problem}

\begin{solution}
  \textsl{LG}1:\\
  $~~~~base \gets m;~k \gets 0;$\\
  $~~~~\text{while } base < n \text{ do the following:}$\\
  $~~~~~~~~base \gets base \times m;$\\
  $~~~~~~~~k \gets k + 1;$\\
  显然上述循环会进行$\text{log}_{m}n$次,因此时间复杂度为$O(\text{log}_{m}n)$;\\
  算法执行过程中只定义了4个变量,因此空间复杂度为$O(1)$。
\end{solution}
%%%%%%%%%%%%%%%%%%%%
%\begincorrection	% begin the ``correction'' part (Optional)

%%%%%%%%%%
%\begin{problem}[题号]
%  题目。
%\end{problem}

%\begin{cause}
%  简述错误原因(可选)。
%\end{cause}

% Or use the ``solution'' environment
%\begin{revision}
%  正确解答。
%\end{revision}
%%%%%%%%%%
%%%%%%%%%%%%%%%%%%%%
%\beginfb	% begin the feedback section (Optional)

%你可以写:
%\begin{itemize}
%  \item 对课程及教师的建议与意见
%  \item 教材中不理解的内容
%  \item 希望深入了解的内容
%  \item 等
%\end{itemize}
%%%%%%%%%%%%%%%%%%%%
\end{document}
%%%%%%%%%%%%%%%%%%%%%%%%%%%%%%%%%%%%%%%%%%%%%%%%%%%%%%%%%%%%
% File: hw.tex 						   %
% Description: LaTeX template for homework.                %
%
% Feel free to modify it (mainly the 'preamble' file).     %
% Contact hfwei@nju.edu.cn (Hengfeng Wei) for suggestions. %
%%%%%%%%%%%%%%%%%%%%%%%%%%%%%%%%%%%%%%%%%%%%%%%%%%%%%%%%%%%%

%%%%%%%%%%%%%%%%%%%%%%%%%%%%%%%%%%%%%%%%%%%%%%%%%%%%%%%%%%%%%%%%%%%%%%
% IMPORTANT NOTE: Compile this file using 'XeLaTeX' (not 'PDFLaTeX') %
%
% If you are using TeXLive 2016 on windows,                          %
% you may need to check the following post:                          %
% https://tex.stackexchange.com/q/325278/23098                       %
%%%%%%%%%%%%%%%%%%%%%%%%%%%%%%%%%%%%%%%%%%%%%%%%%%%%%%%%%%%%%%%%%%%%%%

\documentclass[11pt, a4paper, UTF8]{ctexart}
\input{preamble}  % modify this file if necessary

%%%%%%%%%%%%%%%%%%%%
\title{第一讲:算法正确性}
\me{廖玺然}{171860647}{计算机科学与技术系}
\date{\today}     % you can specify a date like ``2017年9月30日''.
%%%%%%%%%%%%%%%%%%%%
\begin{document}
\maketitle
%%%%%%%%%%%%%%%%%%%%
\noplagiarism	% always keep this
%%%%%%%%%%%%%%%%%%%%
\beginthishw	% begin ``this homework (hw)'' part

%%%%%%%%%%
\begin{problem}[DH: 5.6]	% NOTE: use '[]' (instead of '()' or '{}') to provide additional information
  We have seen in the correctness demonstration for the \textbf{reverse} 
  algorithm that only three (well-placed) invariants are sufficient for 
  the proof.\\
  (a) How would you generalize the above claim to any algorithm whose structure
  (i.e., the structure of its flowchart) is similar to \textbf{reverse}?\\
  (b) What kind of flowchart enables a partial correctness proof of the 
  corresponding algorithm with only two invariants, attached to the \textbf{start} 
  and \textbf{stop} points?\\
  (c) How many well-attached invariants are sufficient for proving the partial 
  correctness of an algorithm whose flowchart contains two loops?\\
  (d) For any flowchart with two loops, how many invariants are necessary for 
  proving partial correctness in the method given in the text? How would you 
  classify the sufficient number of assertions according to the structure of 
  a two-loop flowchart? (Hint: consider connectedness, nesting.)
\end{problem}

% The ``remark'' environments (when needed) must be 
% put before the ``solution''/``revision''/``proof'' environments.
%\begin{remark}	% Optional
%  以下解答参考了书籍/网站/讲义 $\ldots$。

%  \noindent 以下解答是与 XXX 同学讨论得到的。
%\end{remark}

\begin{solution}
  (a) An algorithm with one loop needs at least 3 invariants to prove its 
  correctness.\\
  (b) Flowcharts with with only sequential structure.\\
  (c) 4.\\
  (d) 4. Two class: two loops in sequence or one loop nested in another; 
  The nested loops may need more than 4;
\end{solution}
%%%%%%%%%%

%%%%%%%%%%
\begin{problem}[DH: 5.9]
  Construct a function \textbf{equal}($X, Y$) that tests whether the strings 
  $X$ and $Y$ are equal. It should return true or false accordingly. You may 
  use the operations mentioned in the text and those defined above. (Note, 
  however, that there is no way of comparing arbitrarily long strings in a 
  single operation.) Prove the total correctness of your algorithm.
\end{problem}

\begin{remark}
  以下解答参考了课程QQ群内关于“多重条件判断循环是否终止时循环不变量的选择”的问答。
\end{remark}

\begin{solution}
  \textbf{equal}($X, Y$):\\
  $~~~~\text{while}~X \neq \Lambda~\text{and flag == true do the following:}$\\
  $~~~~~~~~\text{flag = true;}$\\
  $~~~~~~~~\text{if}~Y = \Lambda~\text{then flag = false;}$\\
  $~~~~~~~~\text{else if \textbf{eq}(\textbf{head}(X), \textbf{head}(Y)) then}$\\
  $~~~~~~~~~~~~X = \text{tail}(X);$\\
  $~~~~~~~~~~~~Y = \text{tail}(Y);$\\
  $~~~~~~~~\text{else flag = false;}$\\
  $~~~~\text{if}~Y \neq \Lambda~\text{then flag = false;}$\\
  $~~~~\text{return flag;}$\\
  proof.\\
  (1) the start assertion: $X = S1$ and $Y = S2$, $S1$, $S2$ are both strings;\\
  (2) the invariant for the loop: $S1 = S2$ if and only if both $X = Y$ and 
  flag is true;\\
  (3) the end assertion: $S1 = S2$ if and only if flag is true;\\
  (4) convergent: the number of characters in $X$;\\
  partial correctness:\\
  ($1 \rightarrow 2$): for any legal input, (2) always holds, since it's a tautology;\\
  ($2 \rightarrow 3$): if $X = \Lambda$ then, $S1 = S2$ is true if and only if 
  $Y = \Lambda$; if flag is false, then $S1 \neq S2$;\\
  ($2 \rightarrow 2$): if $X \neq \Lambda$ and flag is true, then after carrying 
  out the instructions $X = \text{tail}(X)$ and $Y = \text{tail}(Y)$, the same 
  properties will still hold for the new values of $X$ and $Y$, thus the loop 
  will continue.\\
  total correctness:\\
  every time the loop is executed, the convergent will get the value (convergent - 1), 
  and it cannot be less than 0, thus the algorithm will finally terminate.\\
  considering the proof of partial correctness, total correctness is proved.
\end{solution}
%%%%%%%%%%
% \newpage  % continue in a new page
%%%%%%%%%%
\begin{problem}[DH: 5.10]
  A palindrome is a string that is the same when read forwards and backwards. 
  Consider the following algorithm \textsl{Pal1} for checking whether a string 
  is a palindrome. The algorithm returns true or false, according to whether 
  the input string \textsl{S} is a palindrome or not.\\
  $~~~~Y \gets \text{\textbf{rev}($S$);}$\\
  $~~~~\text{return \textbf{equal}($S, Y$).}$\\
  (a) Prove the total correctness of \textsl{Pal1}.\\
  (b) The termination of \textsl{Pal1} can be easily proved by relying on the 
  termination of the functions \textbf{rev} and \textbf{equal}. Can you generalize 
  this type of reasoning to a similar composition of any two programs?
\end{problem}

% \begin{remark}	
%   Refer to book.
% \end{remark}

\begin{solution}
  (a) partial correctness: \\
  The partial correctness of \textbf{rev} and \textbf{equal} has been proved, 
  and \textbf{rev} returns a string, thus the partial correctness of the 
  algorithm has been proved;\\
  total correctness:\\
  It has been proved that given a string, \textbf{rev} will terminate, with 
  another string returned. And given two strings, \textbf{equal} will terminate, 
  returning whether they are the same. So given a string, the algorithm will 
  finally terminate, plus the partial correctness, the total correctness of 
  the algorithm has been proved.\\
  (b) If the output of one function could be part of legal input of another 
  function, and they both terminate, then the new-constructed structure terminates.  
\end{solution}
%%%%%%%%%%

%%%%%%%%%%%%%
\begin{problem}[DH: 5.12]
  Here is another algorithm, \textsl{Pal2}, designed to perform the same task 
  as \textsl{Pal1}, but more efficiently:\\
  $~~~~X \gets S;$\\
  $~~~~E \gets \text{true};$\\
  $~~~~\text{while}~X \neq \Lambda~\text{do the following:}$\\
  $~~~~~~~~\text{if \textbf{eq}(\textbf{head}($X$),\textbf{last}($X$)) then}~
  X \gets \text{\textbf{all-but-last}(\textbf{tail}($X$))};$\\
  $~~~~~~~~\text{otherwise}~E \gets \text{false}.$\\
  $~~~~\text{return}~E.$\\
  Unfortunately, \textbf{Pal2} is not totally correct. Prove or disprove:\\
  (a) \textsl{Pal2} is partially correct.\\
  (b) \textsl{Pal2} terminates on every input string \textsl{S}.
\end{problem}

\begin{solution}
  (a) It is partially correct.\\
  the invariant for the loop: \textsl{S} is a palindrome if and only if both 
  $X$ is a palindrome and $E$ is true.\\
  If $X = \Lambda$, then \textsl{S} is a palindrome if and only if $E$ is true.\\
  If $X \neq \Lambda$, the loop will continue.\\
  (b) It is not totally correct.\\
  For a non-palindrome, though $E$ is false, $X$ will not get shorter, thus 
  the loop will go on infinitely.
\end{solution}
%%%%%%%%%%

%%%%%%%%%%
\begin{problem}
  证明Euclid算法的部分正确性。
\end{problem}

\begin{remark}
  以下解答参考了博客:最大公约数(gcd):Euclid算法证明及其它\\
  $http://www.cnblogs.com/ider/archive/2010/11/16/gcd_euclid.html$
\end{remark}

\begin{solution}
  引理:对于任意两个整数$m$, $n$, 若$k|m$, $k|n$, 则对任意的$x, y \in \mathbb{Z}$, 
  $k|(xm + yn)$.\\
  令 k=gcd(m,n),则 k|m 并且 k|n;\\
  令 j=gcd(n, m mod n), 则j|n 并且 j|(m mod n);\\
  对于m, 可以用n 表示为 m=pn+(m mod n);\\
  由引理可知 j|m(其中 x=p,y=1), 又 j|n,于是 j 是 m 和 n 的公约数;\\
  因为 k 是 m 和 n 的最大公约数,所以必有 k≥j;\\
  通过另一种表示形式:(m mod n)=m-pn,同理可得:\\
  k|(m mod n),又k|n,于是 k 是 (m mod n) 和 n 的公约数;\\
  同样由 j 是 n 和 (m mod n) 的最大公约数可以得到 j≥k;\\
  综上, k=j,\\
  即gcd(m,n) = gcd(n, m mod n).\\
  (1) 当$m~ mod~ n \neq 0$时,呼叫下一层;\\
  (2) 当$m~ mod~ n = 0$时,n即最大公约数;\\
  得证。
\end{solution}
%%%%%%%%%%%%%%%%%%%%
%\begincorrection	% begin the ``correction'' part (Optional)

%%%%%%%%%%
%\begin{problem}[题号]
%  题目。
%\end{problem}

%\begin{cause}
%  简述错误原因(可选)。
%\end{cause}

% Or use the ``solution'' environment
%\begin{revision}
%  正确解答。
%\end{revision}
%%%%%%%%%%
%%%%%%%%%%%%%%%%%%%%
%\beginfb	% begin the feedback section (Optional)

%你可以写:
%\begin{itemize}
%  \item 对课程及教师的建议与意见
%  \item 教材中不理解的内容
%  \item 希望深入了解的内容
%  \item 等
%\end{itemize}
%%%%%%%%%%%%%%%%%%%%
\end{document}
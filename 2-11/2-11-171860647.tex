%%%%%%%%%%%%%%%%%%%%%%%%%%%%%%%%%%%%%%%%%%%%%%%%%%%%%%%%%%%%
% File: hw.tex 						   %
% Description: LaTeX template for homework.                %
%
% Feel free to modify it (mainly the 'preamble' file).     %
% Contact hfwei@nju.edu.cn (Hengfeng Wei) for suggestions. %
%%%%%%%%%%%%%%%%%%%%%%%%%%%%%%%%%%%%%%%%%%%%%%%%%%%%%%%%%%%%

%%%%%%%%%%%%%%%%%%%%%%%%%%%%%%%%%%%%%%%%%%%%%%%%%%%%%%%%%%%%%%%%%%%%%%
% IMPORTANT NOTE: Compile this file using 'XeLaTeX' (not 'PDFLaTeX') %
%
% If you are using TeXLive 2016 on windows,                          %
% you may need to check the following post:                          %
% https://tex.stackexchange.com/q/325278/23098                       %
%%%%%%%%%%%%%%%%%%%%%%%%%%%%%%%%%%%%%%%%%%%%%%%%%%%%%%%%%%%%%%%%%%%%%%

\documentclass[12pt, a4paper, UTF8]{ctexart}
%%%%%%%%%%%%%%%%%%%%%%%%%%%%%%%%%%%
% File: preamble.tex
%%%%%%%%%%%%%%%%%%%%%%%%%%%%%%%%%%%

\usepackage[top = 1.5cm]{geometry}

% Set fonts commands
\newcommand{\song}{\CJKfamily{song}} 
\newcommand{\hei}{\CJKfamily{hei}} 
\newcommand{\kai}{\CJKfamily{kai}} 
\newcommand{\fs}{\CJKfamily{fs}}

\newcommand{\me}[2]{\author{{\bfseries 姓名:}\underline{#1}\hspace{2em}{\bfseries 学号:}\underline{#2}}}

% Always keep this.
\newcommand{\noplagiarism}{
  \begin{center}
    \fbox{\begin{tabular}{@{}c@{}}
      请独立完成作业,不得抄袭。\\
      若得到他人帮助, 请致谢。\\
      若参考了其它资料,请给出引用。\\
      鼓励讨论,但需独立书写解题过程。
    \end{tabular}}
  \end{center}
}

% Each hw consists of three parts:
% (1) this homework
\newcommand{\beginrequired}{\part{作业 (必做部分)}}
\newcommand{\beginoptional}{\part{作业 (选做部分)}}
% (2) corrections (Optional)
\newcommand{\begincorrection}{\part{订正}}
% (3) any feedback (Optional)
\newcommand{\beginfb}{\part{反馈}}

% For math
\usepackage{amsmath}
\usepackage{amsfonts}
\usepackage{amssymb}

% Define theorem-like environments
\usepackage[amsmath, thmmarks]{ntheorem}

\theoremstyle{break}
\theorembodyfont{\song}
\theoremseparator{}
\newtheorem*{problem}{题目}

\theorempreskip{2.0\topsep}
\theoremheaderfont{\kai\bfseries}
\theoremseparator{:}
% \newtheorem*{remark}{注}
\theorempostwork{\bigskip\hrule}
\newtheorem*{solution}{解答}
\theorempostwork{\bigskip\hrule}
\newtheorem*{revision}{订正}

\theoremstyle{plain}
\newtheorem*{cause}{错因分析}
\newtheorem*{remark}{注}

\theoremstyle{break}
\theorempostwork{\bigskip\hrule}
\theoremsymbol{\ensuremath{\Box}}
\newtheorem*{proof}{证明}

\renewcommand\figurename{图}
\renewcommand\tablename{表}

% For figures
% for fig with caption: #1: width/size; #2: fig file; #3: fig caption
\newcommand{\fig}[3]{
  \begin{figure}[htp]
    \centering
      \includegraphics[#1]{#2}
      \caption{#3}
  \end{figure}
}

% for fig without caption: #1: width/size; #2: fig file
\newcommand{\fignocaption}[2]{
  \begin{figure}[htp]
    \centering
    \includegraphics[#1]{#2}
  \end{figure}
}  % modify this file if necessary

%%%%%%%%%%%%%%%%%%%%
\title{第十一讲:堆与堆排序}
\me{廖玺然}{171860647}
\date{2018 年 5 月 19 日}     % you can specify a date like ``2017年9月30日''.
%%%%%%%%%%%%%%%%%%%%
\begin{document}
\maketitle
%%%%%%%%%%%%%%%%%%%%
\noplagiarism	% always keep this
%%%%%%%%%%%%%%%%%%%%
\beginrequired	% begin ``this homework (hw)'' part

%%%%%%%%%%
\begin{problem}[TC: 6.1-2]	% NOTE: use '[]' (instead of '()' or '{}') to provide additional information
  Show that $n$-element heap has height $\lfloor\lg n\rfloor$.
\end{problem}

% The ``remark'' environments (when needed) must be 
% put before the ``solution''/``revision''/``proof'' environments.
%\begin{remark}	% Optional
%  以下解答参考了书籍/网站/讲义 $\ldots$。

%  \noindent 以下解答是与 XXX 同学讨论得到的。
%\end{remark}

\begin{solution}
  For a heap of height $h$, it has at least $2^{h}$ elements, and at 
  most $2^{h+1} - 1$ elements. Thus if $n\in [2^{h},2^{h+1}-1]$, then 
  the height of the heap would be $\lfloor\lg n\rfloor$.
\end{solution}
%%%%%%%%%%

%%%%%%%%%%
\begin{problem}[TC: 6.1-4]
  Where in a max-heap might the smallest element reside, assuming that 
  all elements are distinct?
\end{problem}

\begin{solution}
  For a max-heap of $n$ element, the smallest element would be among 
  the elements whose indexes $\geq \lfloor n/2\rfloor + 1$.
\end{solution}
%%%%%%%%%%
% \newpage  % continue in a new page
%%%%%%%%%%
\begin{problem}[TC: 6.1-7]
  Show that, with the array representation for storing an $n$-element 
  heap, the leaves are the nodes indexed by $\lfloor n/2\rfloor+1,\lfloor n/2\rfloor+2,\ldots,n$.
\end{problem}

% \begin{remark}	
%   Refer to book.
% \end{remark}

\begin{solution}
  The array storing a heap is always compact, that is, the heap is 
  always filled from left to right and then to the next level. So the 
  last element in the array that has a child is indexed by $\lfloor n/2\rfloor$. 
  This suggests that the leaves are nodes indexed by $\lfloor n/2\rfloor+1,\lfloor n/2\rfloor+2,\ldots,n$.
\end{solution}
%%%%%%%%%%

%%%%%%%%%%
\begin{problem}[TC: 6.2-2]
  Starting with the procedure MAX-HEAPIFY, write pseudocode for the 
  procedure MIN-HEAPIFY($A,i$), which performs the corresponding 
  manipulation on a min-heap. How does the running time of MIN-HEAPIFY 
  compare to that of MAX-HEAPIFY?
\end{problem}

\begin{solution}
  MIN-HEAPIFY($A,i$)\\
  \indent $l$ = LEFT($i$)\\
  \indent $r$ = RIGHT($i$)\\
  \indent if $l\leq A.heap$-$size$ and $A[l] < A[i]$\\
  \indent\indent $smallest$ = $l$\\
  \indent else $smallest$ = $i$\\
  \indent if $r\leq A.heap$-$size$ and $A[r] < A[smallest]$\\
  \indent\indent $smallest$ = $r$\\
  \indent if $smallest\neq i$\\
  \indent\indent exchange $A[i]$ with $A[smallest]$\\
  \indent\indent MIN-HEAPIFY($A,smallest$)\\
  The running time is the same as MAX-HEAPIFY, 'cuz it just changes the 
  comparing order and a variable name.
\end{solution}
%%%%%%%%%

%%%%%%%%%
\begin{problem}[TC: 6.2-5]
  The code for MAX-HEAPIFY is quite efficient in terms of constant 
  factors, except possibly for the recursive call in line 10, which 
  might cause some compilers to produce inefficient code. Write an 
  efficient MAX-HEAPIFY that uses an iterative control construct 
  instead of recursion.
\end{problem}

\begin{solution}
  ITERATIVE-MAX-HEAPIFY($A,i$)\\
  \indent while \textbf{true}\\
  \indent\indent $l$ = LEFT($i$)\\
  \indent\indent $r$ = RIGHT($i$)\\
  \indent\indent if $l\leq A.heap$-$size$ and $A[l] > A[i]$\\
  \indent\indent\indent $largest$ = $l$\\
  \indent\indent else $largest$ = $i$\\
  \indent\indent if $r\leq A.heap$-$size$ and $A[r] > A[largest]$\\
  \indent\indent\indent $largest$ = $r$\\
  \indent\indent if $largest$ == $i$\\
  \indent\indent\indent return\\
  \indent\indent exchange $A[i]$ with $A[largest]$\\
  \indent\indent $i$ = $largest$
\end{solution}
%%%%%%%%%

%%%%%%%%%
\begin{problem}[TC: 6.2-6]
  Show that the worst-case running time of MAX-HEAPIFY on a heap size 
  of size $n$ is $\Omega(\lg n)$.
\end{problem}

\begin{solution}
  A heap of size $n$ has $\lfloor\lg n\rfloor + 1$ levels. If we put 
  the smallest element at the root, we need to call MAX-HEAPIFY at 
  least $\lfloor\lg n\rfloor - 1$ times to push the smallest element 
  to one of the leaves, so the worst-case running time would be $\Omega(\lg n)$.
\end{solution}
%%%%%%%%%

%%%%%%%%%
\begin{problem}[TC: 6.3-3]
  Show that there are at most $\lceil n/2^{h+1}\rceil$ nodes of height 
  $h$ in any $n$-element heap.
\end{problem}

\begin{solution}
  For $h = 0$, the number of nodes is the number of leaves, which is 
  $\lceil n/2\rceil = \lceil n/2^{0 + 1}\rceil$;\\
  Suppose the property holds for nodes of height $k-1$, namely, there 
  are $\lceil n/2^{k-1+1}\rceil$ nodes of height $k - 1$. If we remove 
  all the leaves, there will be $n-\lceil n/2\rceil = \lfloor n/2\rfloor$ 
  nodes left, and we can find another property for height $k$: there 
  are $\lceil\lfloor n/2\rfloor/2^{k-1+1}\rceil$ nodes of height $k$.\\
  Since $\lceil\lfloor n/2\rfloor/2^{k-1+1}\rceil \leq \lceil(n/2)/2^{k-1+1}\rceil 
  = \lceil n/2^{k+1}\rceil$, we know there are at most $\lceil n/2^{k+1}\rceil$ 
  nodes of height $k$;\\
  In conclusion, there are at most $\lceil n/2^{h+1}\rceil$ nodes of 
  height $h$ in a heap.
\end{solution}
%%%%%%%%%

%%%%%%%%%
\begin{problem}[TC: 6.4-2]
  Argue the correctness of HEAPSORT using the following loop invariant:
  \begin{itemize}
    \item [] At the start of each iteration of the \textbf{for} loop of 
    line 2-5, the subarray $A[1..i]$ is a max-heap containing the $i$ 
    smallest elements of $A[1..n]$, and the subarray $A[i+1..n]$ 
    contains the $n-i$ largest elements of $A[1..n]$, sorted.
  \end{itemize}
\end{problem}

\begin{solution}
  \textbf{Initialization} The subarray $A[i+1..n]$ is empty, so the 
  invariant holds.\\
  \textbf{Maintenance} $A[1]$ is the largest element of subarray $A[1..i]$ 
  and is smaller than any element in $A[i+1..n]$. After we swap it 
  with $A[i]$, $A[i..n]$ will contain the $n-i+1$ largest elements, 
  sorted. Decreasing the heap size and calling MAX-HEAPIFY turns 
  $A[1..i-1]$ into a max-heap. Decrementing $i$ sets up the invariant 
  for the next iteration.\\
  \textbf{Termination} After the loop, $i = 1$. This shows $A[2..n]$ 
  is sorted and $A[1]$ is the smallest element of the array, which 
  means the array is sorted.\\
\end{solution}
%%%%%%%%

%%%%%%%%
\begin{problem}[TC: 6.4-4]
  Show that the worst-case running time of HEAPSORT is $\Omega(n\lg n)$.
\end{problem}

\begin{solution}
  $T(n) \geq n + \sum\limits_{i = 1}^{n - 1}\lg i = n + \lg((n-1)!)$,\\
  Since $\lg((n-1)!) = \Theta(n\lg n)$,\\
  $T(n) = \Omega(n\lg n)$.
\end{solution}
%%%%%%%%

%%%%%%%%
\begin{problem}[TC: 6.5-5]
  Argue the correctness of HEAP-INCREASE-KEY using the following loop 
  invariant:
  \begin{itemize}
    \item [] At the start of each iteration of the \textbf{while} loop 
    of line 4-6, the subarray $A[1..A.heap\text{-}size]$ satisfies the 
    max-heap property, except that there may be one violation: $A[i]$ 
    may be larger than $A[\text{PARENT}(i)]$.
  \end{itemize}
  You may assume that the subarray $A[1..A.heap\text{-}size]$ satisfies 
  the max-heap property at the time HEAP-INCREASE-KEY is called.
\end{problem}

\begin{solution}
  \textbf{Initialization} Since $A$ is initially a max-heap, $\forall 
  j,~j\neq i, A[j] < A[\text{PARENT}(j)]$. Since $A[i]$ is increased, 
  it is possible that $A[i] > A[\text{PARENT}(i)]$. If $A[i] < A[\text{PARENT}(i)]$, 
  too, the loop will not be entered, and the preocedure will terminate.\\
  \textbf{Maintenance} When we exchange $A[i]$ with its parent, the 
  max-heap property is satisfied except that $A[\text{PARENT}(i)]$ may 
  be larger than its parent now. Changing $i$ to its parent maintains 
  the invariant.\\
  \textbf{Termination} The loop terminates when $i = 1$ or the max-heap 
  property for $A[i]$ and its parent is satisfied. At the loop termination, 
  $A$ is a max-heap.
\end{solution}
%%%%%%%

%%%%%%%
\begin{problem}[TC: 6.5-7]
  Show how to implement a first-in, first-out queue with a priority 
  queue. Show how to implement a stack with a priority queue.
\end{problem}

\begin{solution}
  Using a max-heap, for a queue we add elements in a decreasing order;\\
  For a stack we add elements in an increasing order.
\end{solution}

\begin{problem}[TC: 6.5-9]
  Give an $O(n\lg k)$-time algorithm to merge $k$ sorted lists into one 
  sorted list, where $n$ is the total number of elements in all the 
  input lists.
\end{problem}

\begin{solution}
  Build a min-heap whoses nodes contain two value: an element from a 
  list, which is the key, and the list where it comes from.\\
  1) Take the minimum element from each list to build such a heap.\\
  2) Take the minimum-key node from the heap, add it to the tail the 
  final list, and then insert the closest larger element in the list it 
  comes from (if there exists such an element) to to the heap.\\
  3) Continue this until the heap is empty.
\end{solution}
%%%%%%%%%%%%%%%%%%%%
%\begincorrection	% begin the ``correction'' part (Optional)

%%%%%%%%%%
\end{document}
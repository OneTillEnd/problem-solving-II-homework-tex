%%%%%%%%%%%%%%%%%%%%%%%%%%%%%%%%%%%%%%%%%%%%%%%%%%%%%%%%%%%%
% File: hw.tex 						   %
% Description: LaTeX template for homework.                %
%
% Feel free to modify it (mainly the 'preamble' file).     %
% Contact hfwei@nju.edu.cn (Hengfeng Wei) for suggestions. %
%%%%%%%%%%%%%%%%%%%%%%%%%%%%%%%%%%%%%%%%%%%%%%%%%%%%%%%%%%%%

%%%%%%%%%%%%%%%%%%%%%%%%%%%%%%%%%%%%%%%%%%%%%%%%%%%%%%%%%%%%%%%%%%%%%%
% IMPORTANT NOTE: Compile this file using 'XeLaTeX' (not 'PDFLaTeX') %
%
% If you are using TeXLive 2016 on windows,                          %
% you may need to check the following post:                          %
% https://tex.stackexchange.com/q/325278/23098                       %
%%%%%%%%%%%%%%%%%%%%%%%%%%%%%%%%%%%%%%%%%%%%%%%%%%%%%%%%%%%%%%%%%%%%%%

\documentclass[12pt, a4paper, UTF8]{ctexart}
%%%%%%%%%%%%%%%%%%%%%%%%%%%%%%%%%%%
% File: preamble.tex
%%%%%%%%%%%%%%%%%%%%%%%%%%%%%%%%%%%

\usepackage[top = 1.5cm]{geometry}

% Set fonts commands
\newcommand{\song}{\CJKfamily{song}} 
\newcommand{\hei}{\CJKfamily{hei}} 
\newcommand{\kai}{\CJKfamily{kai}} 
\newcommand{\fs}{\CJKfamily{fs}}

\newcommand{\me}[2]{\author{{\bfseries 姓名:}\underline{#1}\hspace{2em}{\bfseries 学号:}\underline{#2}}}

% Always keep this.
\newcommand{\noplagiarism}{
  \begin{center}
    \fbox{\begin{tabular}{@{}c@{}}
      请独立完成作业,不得抄袭。\\
      若得到他人帮助, 请致谢。\\
      若参考了其它资料,请给出引用。\\
      鼓励讨论,但需独立书写解题过程。
    \end{tabular}}
  \end{center}
}

% Each hw consists of three parts:
% (1) this homework
\newcommand{\beginrequired}{\part{作业 (必做部分)}}
\newcommand{\beginoptional}{\part{作业 (选做部分)}}
% (2) corrections (Optional)
\newcommand{\begincorrection}{\part{订正}}
% (3) any feedback (Optional)
\newcommand{\beginfb}{\part{反馈}}

% For math
\usepackage{amsmath}
\usepackage{amsfonts}
\usepackage{amssymb}

% Define theorem-like environments
\usepackage[amsmath, thmmarks]{ntheorem}

\theoremstyle{break}
\theorembodyfont{\song}
\theoremseparator{}
\newtheorem*{problem}{题目}

\theorempreskip{2.0\topsep}
\theoremheaderfont{\kai\bfseries}
\theoremseparator{:}
% \newtheorem*{remark}{注}
\theorempostwork{\bigskip\hrule}
\newtheorem*{solution}{解答}
\theorempostwork{\bigskip\hrule}
\newtheorem*{revision}{订正}

\theoremstyle{plain}
\newtheorem*{cause}{错因分析}
\newtheorem*{remark}{注}

\theoremstyle{break}
\theorempostwork{\bigskip\hrule}
\theoremsymbol{\ensuremath{\Box}}
\newtheorem*{proof}{证明}

\renewcommand\figurename{图}
\renewcommand\tablename{表}

% For figures
% for fig with caption: #1: width/size; #2: fig file; #3: fig caption
\newcommand{\fig}[3]{
  \begin{figure}[htp]
    \centering
      \includegraphics[#1]{#2}
      \caption{#3}
  \end{figure}
}

% for fig without caption: #1: width/size; #2: fig file
\newcommand{\fignocaption}[2]{
  \begin{figure}[htp]
    \centering
    \includegraphics[#1]{#2}
  \end{figure}
}  % modify this file if necessary

%%%%%%%%%%%%%%%%%%%%
\title{第八讲:概率分析与随机算法}
\me{廖玺然}{171860647}
\date{\today}     % you can specify a date like ``2017年9月30日''.
%%%%%%%%%%%%%%%%%%%%
\begin{document}
\maketitle
%%%%%%%%%%%%%%%%%%%%
\noplagiarism	% always keep this
%%%%%%%%%%%%%%%%%%%%
\beginrequired	% begin ``this homework (hw)'' part

%%%%%%%%%%
\begin{problem}[CS: 5.6.4]	% NOTE: use '[]' (instead of '()' or '{}') to provide additional information
  In a card game, you remove the jacks, queens, kings and aces from an ordinary 
  deck of cards and shuffle them. You draw a card. If it is an ace, you are paid 
  $\$1.00$, and the game is repeated. If it is a jack, you are paid $\$2.00$, and 
  the game ends. If it is a queen, you are paid $\$3.00$, and the game ends. If 
  it is a king, you are paid $\$4.00$, and the game ends. What is the maximum 
  amount of money a rational person would pay to play the game?
\end{problem}

% The ``remark'' environments (when needed) must be 
% put before the ``solution''/``revision''/``proof'' environments.

\begin{solution}
  Denote $X$ as the amount of money you get after a round of game.\\
  $P(X = 2) = \frac{1}{4};~~P(X = 3) = \frac{1}{4}\times\frac{1}{4} + \frac{1}{4} = \frac{5}{16};$\\
  $P(X = 4) = \frac{1}{4} + \frac{1}{4}\times\frac{1}{4} + \frac{1}{4}\times\frac{1}{4}\times\frac{1}{4} = \frac{21}{64};$\\
  $P(X = k) = (\frac{1}{4})^{k-4}\times\frac{1}{4} + (\frac{1}{4})^{k-3}\times\frac{1}{4} + (\frac{1}{4})^{k-2}\times\frac{1}{4}$\\
  $~~~~~~~~~~~~~~= (\frac{1}{4})^{k-1} + (\frac{1}{4})^{k-2} + (\frac{1}{4})^{k-3}.~~(k \geq 4)$
  \begin{align}
    E(X) & = \sum_{k=2}^{+\infty} P(X = k)\cdot k \notag \\
         & = \frac{1}{2} + \frac{15}{16} + \sum_{k=4}^{+\infty} k\cdot[(\frac{1}{4})^{k-1} + (\frac{1}{4})^{k-2} + (\frac{1}{4})^{k-3}] \notag \\
         & = \frac{23}{16} + \lim_{n\rightarrow+\infty} (\frac{91}{48} - \frac{7}{48}\times(\frac{1}{4})^{n-1} - 28n(\frac{1}{4})^{n}) \notag \\
         & = \frac{10}{3} \notag
  \end{align}
  So a rational person would pay at most $\$3.33$ for the game.
\end{solution}
%%%%%%%%%%

%%%%%%%%%%
\begin{problem}[CS: 5.6.8]
  Prove Theorem 5.23.
\end{problem}

\begin{solution}
  \begin{align}
      \sum_{i=1}^{n} E(X|F_{i})P(F_{i}) & = \sum_{i=1}^{n} [\sum_{j=1}^{k} x_{j}P((X=x_{j})|F_{i})]P(F_{i}) \notag \\
             & = \sum_{i=1}^{n}\sum_{j=1}^{k} x_{j}P((X=x_{j})|F_{i})P(F_{i}) \notag \\
             & = \sum_{i=1}^{n}\sum_{j=1}^{k} x_{j}P((X=x_{j})\cap F_{i}) \notag \\
             & = \sum_{j=1}^{k} x_{j}[\sum_{i=1}^{n} P((X=x_{j})\cap F_{i})] \notag \\
             & = \sum_{j=1}^{k} x_{j}P((X=x_{j})\cap S) \notag \\
             & = \sum_{j=1}^{k} x_{j}P(X=x_{j}) \notag \\
             & = E(X) \notag
  \end{align}
\end{solution}
%%%%%%%%%%
% \newpage  % continue in a new page
%%%%%%%%%%
\begin{problem}[CS: 5.7.2]
  In Problem 1, let $X_{i}$ be the number of correct answers the student gets on 
  Question $i$, that is, $X_{i}$ is either 0 or 1. What is the expected value of 
  $X_{i}$? What is the variance of $X_{i}$? How does the sum of the variances of 
  $X_{1}$ through $X_{5}$ relate to the variance of $X$ for Problem 1?
\end{problem}

% \begin{remark}	
%   Refer to book.
% \end{remark}

\begin{solution}
  $\forall i \in \{ 1,2,3,4,5 \},~E(X_{i}) = 0.6\times 1 + 0.4\times 0 = 0.6;$\\
  $\forall i \in \{ 1,2,3,4,5 \},~V(X_{i}) = 0.6\times(1-0.6)^{2} + 0.4\times(0-0.6)^{2} = 0.24;$\\
  $\sum V(X_{i}) = 5\times 0.24 = 1.2 = V(X).$
\end{solution}
%%%%%%%%%%

%%%%%%%%%%
\begin{problem}[CS: 5.7.4]
  If the quiz in Problem 1 has 100 questions, what is the expected number of right 
  answers, the variance of the expected number of right answers, and the standard 
  deviation of the number of right answers?
\end{problem}

\begin{solution}
  $E(X) = 100 \times 0.6 = 60;$\\
  $V(X) = 100 \times (0.6\times(1-0.6)^{2} + 0.4\times(0-0.6)^{2}) = 24;$\\
  $S(X) = (V(X))^{\frac{1}{2}} = 2\sqrt{6}.$
\end{solution}
%%%%%%%%%%

%%%%%%%%%
\begin{problem}[CS: 5.7.6]
  What is the variance of the number of right answers for someone who knows $80\%$ 
  of the material on which a 25-question quiz is based? What if the quiz has 100 
  questions? 400 questions? How can you ``correct'' these variances for the fact 
  that the ``spread'' in the histogram for the ``number of the right answers'' 
  random variable only doubled when the number of questions in a test was quadrupled?
\end{problem}

\begin{solution}
  $V(X,25) = 25\times(0.8\times(1-0.8)^{2} + 0.2\times(0-0.8)^{2}) = 4;$\\
  $V(X,100) = 100\times(0.8\times(1-0.8)^{2} + 0.2\times(0-0.8)^{2}) = 16;$\\
  $V(X,400) = 400\times(0.8\times(1-0.8)^{2} + 0.2\times(0-0.8)^{2}) = 64.$\\
  The ``spread'' in the histogram is more related to the standard deviation, 
  namely the square root of the variance, then from the examples above, we can 
  find the doubled relationship between ``spread'' and number of questions.
\end{solution}
%%%%%%%%%

%%%%%%%%%
\begin{problem}[CS: 5.7.12]
  How many questions need to be on a short-answer test for you to be $95\%$ sure 
  that someone who knows $80\%$ of the course material gets a grade between $75\%$ 
  and $85\%$?
\end{problem}

\begin{solution}
  $V(X) = n\times(0.8\times(1-0.8)^{2} + 0.2\times(0-0.8)^{2}) = 0.16n;$\\
  $S(X) = 0.4\sqrt{n};$\\
  Let $2S(X) = 0.05n$, we have $n = 256.$
\end{solution}
%%%%%%%%

%%%%%%%%
\begin{problem}[TC: 5.2-4]
  Use indicator random variables to solve the following problem, which is known 
  as the \textsl{\textbf{hat-check problem}}. Each of $n$ customers gives a hat 
  to a hat-check person at a restaurant. The hat-check person gives the hat back 
  to the customers in a random order. What is the expected number of customers 
  who get back their own hat?
\end{problem}

\begin{solution}
  Let $X_{i} = I\{\text{the $i$-th customer gets his own hat}\}$, and $X = X_{1} + X_{2} + \cdots + X_{n}$, 
  then
  \begin{align}
      E(X) & = E[\sum_{i=1}^{n} X_{i}] \notag \\
           & = \sum_{i=1}^{n} E[X_{i}] \notag \\
           & = \sum_{i=1}^{n} \frac{\binom{n-1}{i-1}}{\binom{n}{i-1}}\cdot\frac{1}{n-i+1} \notag \\
           & = \sum_{i=1}^{n} \frac{1}{n} \notag \\
           & = 1 \notag
  \end{align}
\end{solution}
%%%%%%%%%

%%%%%%%%%
\begin{problem}[TC: 5.2-5]
  Let $A[1..n]$ be an array of $n$ distinct numbers. If $i < j$ and $A[i] > A[j]$, 
  then the pair $(i,j)$ is called an \textbf{\textsl{inversion}} of $A$. Suppose 
  that the elements of $A$ form a uniform random permutation of $\langle 1,2,...,n \rangle$. 
  Use indicator random variables to compute the expected number of inversions.
\end{problem}

\begin{solution}
  Let $X_{ij} = I\{\text{$(i,j)$ is an inversion}\}$,
  \[X = \sum_{i=1}^{n-1}\sum_{j=j+1}^{n} X_{ij},\]
  then
  \begin{align}
      E(X) & = E[\sum_{i=1}^{n-1}\sum_{j=j+1}^{n} X_{ij}] \notag \\
           & = \sum_{i=1}^{n-1}\sum_{j=j+1}^{n} E[X_{ij}] \notag \\
           & = \sum_{i=1}^{n-1}\sum_{j=j+1}^{n} \frac{1}{2} \notag \\
           & = \frac{1}{2} \times \frac{n(n-1)}{2} \notag \\
           & = \frac{n(n-1)}{4} \notag
  \end{align}
\end{solution}
%%%%%%%%%

%%%%%%%%%
\begin{problem}[TC: 5.3-2]
  Professor Kelp decides to write a procedure that produces at random any permutation 
  besides the identity permutation. Does the code do what Professor Kelp intends?
\end{problem}

\begin{solution}
  No, this code can never generate permutations whose first element is the first 
  element of $A$.
\end{solution}
%%%%%%%%%

%%%%%%%%%%
\begin{problem}[TC: 5.3-3]
  Suppose that instead of swapping element $A[i]$ with a random element from the 
  subarray $A[i..n]$, we swapped it with a random element from anywhere in the 
  array. Does the code produce a uniform random permutation? Why or why not?
\end{problem}

\begin{solution}
  No, this code would generate at most $n^{n}$ permutations (including the same 
  ones), and $n^{n}$ is not divisible by $n!$, so this code cannot generate ``uniform'' 
  random permutations.
\end{solution}
%%%%%%%%

%%%%%%%
\begin{problem}[TC: 5.3-4]
  Professor Armstrong suggests a procedure for generating a uniform random permutation. 
  Show that each element $A[i]$ has a $1/n$ probability of winding up in any particular 
  position in $B$. Then show that Professor Armstrong is mistaken by showing that 
  the result permutation is not uniform random.
\end{problem}

\begin{solution}
  $\forall k \in \{ 1,2,...,n \},~P(offset = k) = 1/n$, which means $\forall A[i]$, 
  it has the same probability $1/n$ to be the $((i + offset)\text{ mod }n)$-th 
  element in $B$, that is, it has $1/n$ probability of winding up in any position 
  in $B$.\\
  This code just generate a series of circulated sequences, and there are at most 
  $n$ permutations, so they are not ``uniform random'', Professor is mistaken.
\end{solution}
%%%%%%%%%%%%%%%%%%%%
\end{document}
%%%%%%%%%%%%%%%%%%%%%%%%%%%%%%%%%%%%%%%%%%%%%%%%%%%%%%%%%%%%
% File: hw.tex 						   %
% Description: LaTeX template for homework.                %
%
% Feel free to modify it (mainly the 'preamble' file).     %
% Contact hfwei@nju.edu.cn (Hengfeng Wei) for suggestions. %
%%%%%%%%%%%%%%%%%%%%%%%%%%%%%%%%%%%%%%%%%%%%%%%%%%%%%%%%%%%%

%%%%%%%%%%%%%%%%%%%%%%%%%%%%%%%%%%%%%%%%%%%%%%%%%%%%%%%%%%%%%%%%%%%%%%
% IMPORTANT NOTE: Compile this file using 'XeLaTeX' (not 'PDFLaTeX') %
%
% If you are using TeXLive 2016 on windows,                          %
% you may need to check the following post:                          %
% https://tex.stackexchange.com/q/325278/23098                       %
%%%%%%%%%%%%%%%%%%%%%%%%%%%%%%%%%%%%%%%%%%%%%%%%%%%%%%%%%%%%%%%%%%%%%%

\documentclass[12pt, a4paper, UTF8]{ctexart}
%%%%%%%%%%%%%%%%%%%%%%%%%%%%%%%%%%%
% File: preamble.tex
%%%%%%%%%%%%%%%%%%%%%%%%%%%%%%%%%%%

\usepackage[top = 1.5cm]{geometry}

% Set fonts commands
\newcommand{\song}{\CJKfamily{song}} 
\newcommand{\hei}{\CJKfamily{hei}} 
\newcommand{\kai}{\CJKfamily{kai}} 
\newcommand{\fs}{\CJKfamily{fs}}

\newcommand{\me}[2]{\author{{\bfseries 姓名:}\underline{#1}\hspace{2em}{\bfseries 学号:}\underline{#2}}}

% Always keep this.
\newcommand{\noplagiarism}{
  \begin{center}
    \fbox{\begin{tabular}{@{}c@{}}
      请独立完成作业,不得抄袭。\\
      若得到他人帮助, 请致谢。\\
      若参考了其它资料,请给出引用。\\
      鼓励讨论,但需独立书写解题过程。
    \end{tabular}}
  \end{center}
}

% Each hw consists of three parts:
% (1) this homework
\newcommand{\beginrequired}{\part{作业 (必做部分)}}
\newcommand{\beginoptional}{\part{作业 (选做部分)}}
% (2) corrections (Optional)
\newcommand{\begincorrection}{\part{订正}}
% (3) any feedback (Optional)
\newcommand{\beginfb}{\part{反馈}}

% For math
\usepackage{amsmath}
\usepackage{amsfonts}
\usepackage{amssymb}

% Define theorem-like environments
\usepackage[amsmath, thmmarks]{ntheorem}

\theoremstyle{break}
\theorembodyfont{\song}
\theoremseparator{}
\newtheorem*{problem}{题目}

\theorempreskip{2.0\topsep}
\theoremheaderfont{\kai\bfseries}
\theoremseparator{:}
% \newtheorem*{remark}{注}
\theorempostwork{\bigskip\hrule}
\newtheorem*{solution}{解答}
\theorempostwork{\bigskip\hrule}
\newtheorem*{revision}{订正}

\theoremstyle{plain}
\newtheorem*{cause}{错因分析}
\newtheorem*{remark}{注}

\theoremstyle{break}
\theorempostwork{\bigskip\hrule}
\theoremsymbol{\ensuremath{\Box}}
\newtheorem*{proof}{证明}

\renewcommand\figurename{图}
\renewcommand\tablename{表}

% For figures
% for fig with caption: #1: width/size; #2: fig file; #3: fig caption
\newcommand{\fig}[3]{
  \begin{figure}[htp]
    \centering
      \includegraphics[#1]{#2}
      \caption{#3}
  \end{figure}
}

% for fig without caption: #1: width/size; #2: fig file
\newcommand{\fignocaption}[2]{
  \begin{figure}[htp]
    \centering
    \includegraphics[#1]{#2}
  \end{figure}
}  % modify this file if necessary

%%%%%%%%%%%%%%%%%%%%
\title{第四讲:分治法与递归}
\me{廖玺然}{171860647}{计算机科学与技术系}
\date{2018 年 3 月 28 日}     % you can specify a date like ``2017年9月30日''.
%%%%%%%%%%%%%%%%%%%%
\begin{document}
\maketitle
%%%%%%%%%%%%%%%%%%%%
\noplagiarism	% always keep this
%%%%%%%%%%%%%%%%%%%%
\beginthishw	% begin ``this homework (hw)'' part

%%%%%%%%%%
\begin{problem}[TC: exercise 4.1-5]	% NOTE: use '[]' (instead of '()' or '{}') to provide additional information
  Use the following ideas to develop a nonrecursive, linear-time algorithm 
  for the maximum-subarray problem. Start at the left end of the array, and 
  progress toward the right, keeping track of the maximum subarray seen so 
  far. Knowing a maximum array of $A[1..j]$, extend the answer to find a 
  maximum subarray ending at index $j+1$ by using the following observation: 
  a maximum subarray of $A[1..j + 1]$ is either a maximum subarray of $A[1,,j]$ 
  or a subarray $A[i..j + 1]$, for some $1 \leq i \leq j + 1$. Determine a 
  maximum subarray of the form $A[i..j + 1]$ in constant time based on Knowing 
  a maximum subarray ending at index $j$.
\end{problem}

% The ``remark'' environments (when needed) must be 
% put before the ``solution''/``revision''/``proof'' environments.
\begin{remark}	% Optional
  以下解答参考了博客“最大子序列和问题”,网址:\\
  https://blog.csdn.net/lingfeng2019/article/details/78672685
\end{remark}

\begin{solution}
  $\text{int } max = 0,~tmp = 0,~cnt = 0;$\\
  $\text{int } head[n] = \{ 0 \},~tail[n] = \{ 0 \}$\\
  $head[0] = 1;$\\
  $\text{for } i \text{ from } 1 \text{ to } n$\\
  $\indent \text{input } A[i];$\\
  $\indent tmp = A[i] + tmp;$\\
  $\indent \text{if } tmp > max \text{ then}$\\
  $\indent \indent max = tmp;$\\
  $\indent \indent head[0] = head[cnt];$\\
  $\indent \indent cnt = 0;$\\
  $\indent \indent tail[cnt] = i;$\\
  $\indent \text{if } tmp = max \text{ then}$\\
  $\indent \indent tail[cnt] = i;$\\
  $\indent \text{if } tmp < 0 \text{ then}$\\
  $\indent \indent head[++cnt] = i + 1;$\\
  $\text{for } i \text{ from } 0 \text{ to } cnt$\\
  $\indent \text{output } A[head[i],..,tail[i]].$
\end{solution}
%%%%%%%%%%

%%%%%%%%%%
\begin{problem}[TC: exercise 4,3-3]
  We saw that the solution of $T(n) = 2T([n/2]) + n$ is $O(n\text{ lg }n)$. 
  Show that the solution of this recurrence is also $\Omega(n\text{ lg }n)$. 
  Conclude that the solution is $\Theta(n\text{ lg }n)$.
\end{problem}

\begin{solution}
  for $n = 1$, $T(1) = 1 > cn \text{lg} n = 0,$\\
  for $2 \leq n \leq k(k \geq 2)$, assume that $T(k) \geq ck \text{lg} k$, then\\
  for $n = k + 1$
  \begin{align}
      T(k + 1) & = 2T([(k + 1)/2]) + k + 1 \notag \\
               & \geq 2c[(k + 1)/2]\text{lg}([k/2]) + k + 1 \notag \\
               & \geq c(k + 1)\text{lg}((k + 1)/2) + k + 1 \notag \\
               & = c(k + 1)\text{lg}(k + 1) - c(k + 1)\text{lg}2 + k + 1 \notag \\
               & = c(k + 1)\text{lg}(k + 1) + (1 - c)(k + 1) \notag
  \end{align}
  for $0 < c \leq 1$, we have that $T(k + 1) \geq c(k + 1)\text{lg}(k + 1),$\\
  so we can conclude that $T(n) = \Omega(n\text{ lg }n),$\\
  plus the conclusion that $T(n) = O(n\text{ lg }n)$, we find that $T(n) = \Theta(n\text{ lg }n)$.
\end{solution}
%%%%%%%%%%
% \newpage  % continue in a new page
%%%%%%%%%%
\begin{problem}[TC: exercise 4.4-2]
  Use a recursion tree to determine a good asymptotic upper bound on the 
  recurrence $T(n) = T(n/2) + n^{2}$. Use the substitution method to verify 
  your answer.	
\end{problem}

% \begin{remark}	
%   Refer to book.
% \end{remark}

\begin{solution}
  The recursion tree would be like\\
  \fignocaption{width = 1.0\textwidth}{figs/tree_1}
  \begin{align}
  g(n) & = \sum_{i = 0}^{\text{log}_{2}n - 1} \frac{1}{4^{i}}cn^{2} \notag \\
       & = \frac{1 - \frac{1}{4}^{\text{log}_{2}n - 1}}{1 - \frac{1}{1}}cn^{2} \notag \\
       & = \frac{4}{3}cn^{2} - \frac{16}{3}c \notag 
  \end{align}
  we can also use Master theorem to find this
  \begin{align}
      T(n) = O(n^{2}) \notag
  \end{align}
  verification:\\
  for $n = 1$, $T(1) = 1 \leq cn^{2} = 1,$\\
  for $2 \leq n \leq k(k \geq 2)$, assume that $T(k) \leq ck^{2}$, then\\
  for $n = k + 1$
  \begin{align}
      T(k + 1) & = T([(k + 1)/2]) + (k + 1)^{2} \notag \\
               & \leq c[(k + 1)/2]^{2} + (k + 1)^{2} \notag \\
               & \leq \frac{c}{4}(k + 1)^{2} + (k + 1)^{2} \notag \\
               & = (1 + c/4)(k + 1)^{2} \notag
  \end{align}
  for $0 < c \leq \frac{4}{3}$, we have that $T(k + 1) \leq c(k + 1)^{2},$\\
  so we can conclude that $T(n) = O(n^{2}),$\\
\end{solution}
%%%%%%%%%%

%%%%%%%%%%
\begin{problem}[TC: exercise 4.5-3]
  Use the master method to show that the solution to the binary-search recurrence 
  $T(n) = T(n/2) + \Theta(1)$ is $T(n) = \Theta(\text{lg }n)$
\end{problem}

\begin{solution}
  In this case, $a = 1,~b = 2,~f(n) = \Theta(1) = \Theta(n^{\text{log}_{b}a}),$\\
  so $g(n) = \Theta(\text{lg }n)$, then $T(n) = \Theta(\text{lg }n)$.
\end{solution}
%%%%%%%%%%%

%%%%%%%%%%%
\begin{problem}[TC: problem 4.4]
  This problem develops properties of the Fibonacci numbers, which are defined 
  by recurrence (3.22). We shall use the technique of generating functions to 
  solve the Fibonacci recurrence. Define the \textbf{\textsl{generating function}} 
  (or \textbf{\textsl{formal power series}}) $\mathcal{F}$ as
  \begin{align}
      \mathcal{F}(z) & = \sum_{i = 0}^{\infty} F_{i}z^{i} \notag \\
                     & = 0 + z + z^{2} + 2z^{3} + 3z^{4} + 5z^{5} + 8z^{6} 
                     + 13z^{7} + 21z^{8} + \cdots, \notag
  \end{align}
  where $F_{i}$ is the $i$th Fibonacci number.\\
  \textsl{\textbf{a.}} Show that $\mathcal{F}(z) = z + z\mathcal{F}(z) + z^{2}\mathcal{F}(z).$\\
  \textsl{\textbf{b.}} Show that
  \begin{align}
      \mathcal{F}(z) & = \frac{z}{1 - z - z^{2}} \notag \\
                     & = \frac{z}{(1 - \phi z)(1 - \hat{\phi}z)} \notag \\
                     & = \frac{1}{\sqrt{5}}(\frac{1}{1 - \phi z} - \frac{1}{1 - \hat{\phi}z}) \notag
  \end{align}
  $~~~~$where\\
  $~~~~~\phi = \frac{1 + \sqrt{5}}{2}$\\
  $~~~~~$and\\
  $~~~~~\hat{\phi} = \frac{1 - \sqrt{5}}{2}.$\\
  \textsl{\textbf{c.}} Show that
  \[ \mathcal{F}(z) = \sum_{i = 0}^{\infty} \frac{1}{\sqrt{5}}(\phi^{i} - \hat{\phi}^{i})z^{i}. \]
  \textsl{\textbf{d.}} Use part (c) to prove that $F_{i} = \phi^{i} / \sqrt{5}$ 
  for $i > 0$, rounded to the nearest integer. 
\end{problem}

\begin{solution}
  a. 
  \begin{align}
      z \mathcal{F}(z) & = \sum_{i = 2}^{\infty} F_{i - 1}z^{i} \notag \\
      z^{2} \mathcal{F}(z) & = \sum_{i = 3}^{\infty} F_{i - 2}z^{i} \notag \\
      z + z \mathcal{F}(z) + z^{2} \mathcal{F}(z) & = 0 + z + z^{2} 
      + \sum_{i = 3}^{\infty} (F_{i - 1} + F_{i - 2})z^{i} \notag \\
                       & = 0 + z + z^{2} + \sum_{i = 3}^{\infty} F_{i}z^{i} \notag \\
                       & = \sum_{i = 0}^{\infty} F_{i}z^{i} \notag \\
                       & = \mathcal{F}(z) \notag
  \end{align}
  b. since $\mathcal{F}(z) = z + z\mathcal{F}(z) + z^{2}\mathcal{F}(z),$\\
  $~~~~~$we have $(1 - z - z^{2})\mathcal{F}(z) = z,$\\
  $~~~~~$so $\mathcal{F}(z) = \frac{z}{1 - z - z^{2}},$\\
  $~~~~~$since $(1 - \phi z)(1 - \hat{\phi}z) = 1 - (\frac{1 + \sqrt{5}}{2} + \frac{1 - \sqrt{5}}{2})z 
                + \frac{1 - 5}{4}z^{2} = 1 - z - z^{2},$\\
  $~~~~~$and $\frac{1}{\sqrt{5}}(\frac{1}{1 - \phi z} - \frac{1}{1 - \hat{\phi}z}) 
              = \frac{1}{\sqrt{5}} \frac{(\frac{1 + \sqrt{5}}{2} - \frac{1 - \sqrt{5}}{2})z}{(1 - \phi z)(1 - \hat{\phi}z)} 
              = \frac{z}{(1 - \phi z)(1 - \hat{\phi}z)},$\\
  $~~~~~$we can conclude that
  \begin{align}
    \mathcal{F}(z) & = \frac{z}{1 - z - z^{2}} \notag \\
                   & = \frac{z}{(1 - \phi z)(1 - \hat{\phi}z)} \notag \\
                   & = \frac{1}{\sqrt{5}}(\frac{1}{1 - \phi z} - \frac{1}{1 - \hat{\phi}z}) \notag
  \end{align}
  c. for $i \geq 3,~F_{i} = F_{i - 1} + F_{i - 2},$\\
  $~~~~~$assume $F_{i} + kF_{i - 1} = (1 + k)(F_{i - 1} + \frac{1}{1 + k}F_{i - 2}),$\\
  $~~~~~$let $k = \frac{1}{1 + k}$, we have $k = \frac{-1 \pm \sqrt{5}}{2}$, then we have
  \begin{align}
      F_{i} + \frac{-1 + \sqrt{5}}{2}F_{i - 1} & = \frac{1 + \sqrt{5}}{2}(F_{i - 1} + \frac{-1 + \sqrt{5}}{2}F_{i - 2}) \\
      F_{i} + \frac{-1 - \sqrt{5}}{2}F_{i - 1} & = \frac{1 - \sqrt{5}}{2}(F_{i - 1} + \frac{-1 - \sqrt{5}}{2}F_{i - 2}) 
  \end{align}
  $~~~~~$since $F_{1} = 1,~F_{2} = 1$, we have
  \begin{align}
    F_{i} + \frac{-1 + \sqrt{5}}{2}F_{i - 1} & = (\frac{1 + \sqrt{5}}{2})^{i - 1} \\
    F_{i} + \frac{-1 - \sqrt{5}}{2}F_{i - 1} & = (\frac{1 - \sqrt{5}}{2})^{i - 1}    
  \end{align}
  $~~~~~(3) \times \frac{1 + \sqrt{5}}{2} - (4) \times \frac{1 - \sqrt{5}}{2}$, we have
  \begin{align}
      F_{i} & = \frac{1}{\sqrt{5}}((\frac{1 + \sqrt{5}}{2})^{i} - (\frac{1 - \sqrt{5}}{2})^{i}) \notag \\
            & = \frac{1}{\sqrt{5}}(\phi^{i} - \hat{\phi}^{i})
  \end{align}
  $~~~~~$so we can conclude that
  \[ \mathcal{F}(z) = \sum_{i = 0}^{\infty} \frac{1}{\sqrt{5}}(\phi^{i} - \hat{\phi}^{i})z^{i}. \]
  d. Since $|\hat{\phi}| < 1,~|\frac{1}{\sqrt{5}}\hat{\phi}^{i}| < 1/2$,\\
  $~~~~~$So rounded to the nearest integer, $F_{i} = \phi^{i} / \sqrt{5}.$
\end{solution}
%%%%%%%%%%%%%%%%%%%%

%%%%%%%%%%%%%%%%%%%%
\end{document}
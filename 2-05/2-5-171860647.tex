%%%%%%%%%%%%%%%%%%%%%%%%%%%%%%%%%%%%%%%%%%%%%%%%%%%%%%%%%%%%
% File: hw.tex 						   %
% Description: LaTeX template for homework.                %
%
% Feel free to modify it (mainly the 'preamble' file).     %
% Contact hfwei@nju.edu.cn (Hengfeng Wei) for suggestions. %
%%%%%%%%%%%%%%%%%%%%%%%%%%%%%%%%%%%%%%%%%%%%%%%%%%%%%%%%%%%%

%%%%%%%%%%%%%%%%%%%%%%%%%%%%%%%%%%%%%%%%%%%%%%%%%%%%%%%%%%%%%%%%%%%%%%
% IMPORTANT NOTE: Compile this file using 'XeLaTeX' (not 'PDFLaTeX') %
%
% If you are using TeXLive 2016 on windows,                          %
% you may need to check the following post:                          %
% https://tex.stackexchange.com/q/325278/23098                       %
%%%%%%%%%%%%%%%%%%%%%%%%%%%%%%%%%%%%%%%%%%%%%%%%%%%%%%%%%%%%%%%%%%%%%%

\documentclass[11pt, a4paper, UTF8]{ctexart}
%%%%%%%%%%%%%%%%%%%%%%%%%%%%%%%%%%%
% File: preamble.tex
%%%%%%%%%%%%%%%%%%%%%%%%%%%%%%%%%%%

\usepackage[top = 1.5cm]{geometry}

% Set fonts commands
\newcommand{\song}{\CJKfamily{song}} 
\newcommand{\hei}{\CJKfamily{hei}} 
\newcommand{\kai}{\CJKfamily{kai}} 
\newcommand{\fs}{\CJKfamily{fs}}

\newcommand{\me}[2]{\author{{\bfseries 姓名:}\underline{#1}\hspace{2em}{\bfseries 学号:}\underline{#2}}}

% Always keep this.
\newcommand{\noplagiarism}{
  \begin{center}
    \fbox{\begin{tabular}{@{}c@{}}
      请独立完成作业,不得抄袭。\\
      若得到他人帮助, 请致谢。\\
      若参考了其它资料,请给出引用。\\
      鼓励讨论,但需独立书写解题过程。
    \end{tabular}}
  \end{center}
}

% Each hw consists of three parts:
% (1) this homework
\newcommand{\beginrequired}{\part{作业 (必做部分)}}
\newcommand{\beginoptional}{\part{作业 (选做部分)}}
% (2) corrections (Optional)
\newcommand{\begincorrection}{\part{订正}}
% (3) any feedback (Optional)
\newcommand{\beginfb}{\part{反馈}}

% For math
\usepackage{amsmath}
\usepackage{amsfonts}
\usepackage{amssymb}

% Define theorem-like environments
\usepackage[amsmath, thmmarks]{ntheorem}

\theoremstyle{break}
\theorembodyfont{\song}
\theoremseparator{}
\newtheorem*{problem}{题目}

\theorempreskip{2.0\topsep}
\theoremheaderfont{\kai\bfseries}
\theoremseparator{:}
% \newtheorem*{remark}{注}
\theorempostwork{\bigskip\hrule}
\newtheorem*{solution}{解答}
\theorempostwork{\bigskip\hrule}
\newtheorem*{revision}{订正}

\theoremstyle{plain}
\newtheorem*{cause}{错因分析}
\newtheorem*{remark}{注}

\theoremstyle{break}
\theorempostwork{\bigskip\hrule}
\theoremsymbol{\ensuremath{\Box}}
\newtheorem*{proof}{证明}

\renewcommand\figurename{图}
\renewcommand\tablename{表}

% For figures
% for fig with caption: #1: width/size; #2: fig file; #3: fig caption
\newcommand{\fig}[3]{
  \begin{figure}[htp]
    \centering
      \includegraphics[#1]{#2}
      \caption{#3}
  \end{figure}
}

% for fig without caption: #1: width/size; #2: fig file
\newcommand{\fignocaption}[2]{
  \begin{figure}[htp]
    \centering
    \includegraphics[#1]{#2}
  \end{figure}
}  % modify this file if necessary

%%%%%%%%%%%%%%%%%%%%
\title{第五讲:递归及其数学基础}
\me{廖玺然}{171860647}{计算机科学与技术系}
\date{2018 年 4 月 4 日}     % you can specify a date like ``2017年9月30日''.
%%%%%%%%%%%%%%%%%%%%
\begin{document}
\maketitle
%%%%%%%%%%%%%%%%%%%%
\noplagiarism	% always keep this
%%%%%%%%%%%%%%%%%%%%
\beginthishw	% begin ``this homework (hw)'' part

%%%%%%%%%%
\begin{problem}[CS: 4.1.16]	% NOTE: use '[]' (instead of '()' or '{}') to provide additional information
  An alternate version of the Ear Lemma states that a triangulated polygon 
  is either a triangle with three ears or has at least at least two ears.
  (This version does not specify that the ears are nonadjacent.) What happens 
  if we try proving this by induction, using the same decomposition that we 
  used in proving the Ear Lemma?
\end{problem}

% The ``remark'' environments (when needed) must be 
% put before the ``solution''/``revision''/``proof'' environments.

\begin{solution}
  If the original polygon is larger than a triangle, and the two separated 
  polygons are both larger than a triangle, then according to the statement, 
  both of them may have two adjacent ears, and the original diagnal may be 
  incident to all of them, in this case we cannot find any ears in the original 
  polygon, which means the decomposition used before does not work.
\end{solution}
%%%%%%%%%%

%%%%%%%%%%
\begin{problem}[CS: 4.1.17]
  There is a relationship between the number of vertices in a polygon and 
  the number of triangles in any triangulation of that polygon. State this 
  relationship and prove it by induction.
\end{problem}

\begin{solution}
  Denote the number of vertices in a polygon as $n$, the number of triangles 
  in a triangulation of that polygon is $f(n)$, then $f(n) = n - 2.$\\
  Proof.\\
  for $n = 3$, we can easily point out that $f(n) = 1 = n - 2$;\\
  suppose for $3 \leq m < n$, it holds that $f(n) = n - 2$, then for $n$,
  we can split the polygon into two smaller polygons that have $a$ vertices 
  and $b$ vertices respectively, and there are three obvious properties 
  between $a,b,n$:\\
  1) $a < n,~b < n$;\\
  2) $a + b = n + 2$;\\
  3) $f(a) + f(b) = f(n)$;\\
  according to the induction hypothesis, $f(a) = a - 2,~f(b) = b - 2$, so 
  we find that $f(n) = f(a) + f(b) = a + b - 4 = n - 2$,\\
  thus it hols for all $n \geq 3$ that $f(n) = n - 2$. 
\end{solution}
%%%%%%%%%%
% \newpage  % continue in a new page
%%%%%%%%%%
\begin{problem}[CS: 4.2.8]
  At the end of each year, a state fish hatchery puts 2000 fish into a lake. 
  The number of fish in the lake at the beginning of the year doubles by the 
  end of the year due to reproduction. Give a recurrence for the number of 
  fish in the lake after $n$ years, and solve the recurrence.	
\end{problem}

% \begin{remark}	
%   Refer to book.
% \end{remark}

\begin{solution}
  Denote the number of fish after $n$ years as $f(n)~(n \geq 1)$, the number 
  of fish at the start time (the beginning of the year) as $f(0)$, then we 
  have:\\
  $f(n) = 2f(n - 1) + 2000$;\\
  $f(n) + k = 2(f(n - 1) + 1000 + k/2)$;\\
  let $k = 1000 + k/2$, we get that $k = 2000$, then\\
  $f(n) + 2000 = 2(f(n - 1) + 2000)$,\\
  $f(n) + 2000 = 2^{n}(f(0) + 2000)$,\\
  $f(n) = 2^{n}f(0) + 2000(2^{n} - 1)$.
\end{solution}
%%%%%%%%%%

%%%%%%%%%%
\begin{problem}[CS: 4.2.11]
  Solve the recurrence $T(n) = 2T(n - 1) + n2^{n}$, with the initial condition 
  $T(0) = 1$.
\end{problem}

\begin{solution}
  since $T(n) = 2T(n - 1) + n2^{n}$, we have
  \begin{align}
    T(n) & = 2^{n}T(0) + \sum_{i = 1}^{n} 2^{n - i} \cdot i2^{i} \notag \\
         & = 2^{n} + 2^{n} \cdot \frac{n(n + 1)}{2} \notag \\
         & = 2^{n - 1}(n^{2} + n + 2) \notag
  \end{align}
\end{solution}
%%%%%%%%%%%

%%%%%%%%%%
\begin{problem}[CS: 4.5.9]
  Draw recursion trees, and use them to find big $\Theta$ bounds on the solutions 
  to the following recurrences. For each, assume that $T(1) = 1$ and that $n$ 
  is a power of the appropriate integer.\\
  a. $T(n) = 8T(n/2) + n$\\
  b. $T(n) = 8T(n/2) + n^{3}$\\
  c. $T(n) = 3T(n/2) + n$\\
  d. $T(n) = T(n/4) + 1$\\
  e. $T(n) = 3T(n/3) + n^{2}$
\end{problem}

\begin{solution}
  a. \fignocaption{width = 1.0\textwidth}{figs/tree_1}
  $T(n) = n^{3} + n \sum_{i = 0}^{\text{log}_{2}n - 1} 4^{i} = n^{3} + \frac{1}{3} (n^{3} - n) 
  = \Theta (n^{3})$.\\
  b. \fignocaption{width = 1.0\textwidth}{figs/tree_2}
  $T(n) = n^{3} + \sum_{i = 0}^{n = \text{log}_{2}n - 1} n^{3} = n^{3}(\text{log}_{2}n + 1) 
  = \Theta (n^{3}\text{log}_{2}n)$.\\
  c. \fignocaption{width = 0.7\textwidth}{figs/tree_3}
  $T(n) = 3^{\text{log}_{2}n} + n \sum_{i = 0}^{\text{log}_{2}n - 1} (\frac{3}{2})^{i} 
  = n^{\text{log}_{2}3} + 2(n^{\text{log}_{2}3} - n) = \Theta (n^{\text{log}_{2}3})$.\\
  d. \fignocaption{width = 0.7\textwidth}{figs/tree_4}
  $T(n) = 1 + \text{log}_{4}n = 1 + \frac{1}{2} \text{log}_{2}n = \Theta (\text{log}_{2}n)$.\\
  e. \fignocaption{width = 0.7\textwidth}{figs/tree_5}
  $T(n) = n + n^{2}\sum_{i = 0}^{\text{log}_{3}n - 1} \frac{1}{3^{i}} = n + \frac{3}{2}(n^{2} - n) 
  = \Theta (n^{2})$.
\end{solution}
%%%%%%%%%%

%%%%%%%%%%
\begin{problem}[CS: 4.4.1]
  Use the master theorem to give big $\Theta$ bounds on the solutions to 
  the following recurrences. For each, assume that $T(0) = 1$ and that $n$ 
  is a power of the appropriate integer.\\
  a. $T(n) = 8T(n/2) + n$\\
  b. $T(n) = 8T(n/2) + n^{3}$\\
  c. $T(n) = 3T(n/2) + n$\\
  d. $T(n) = T(n/4) + 1$\\
  e. $T(n) = 3T(n/3) + n^{2}$
\end{problem}

\begin{solution}
  a. $\text{log}_{b}a = \text{log}_{2}8 = 3 > 1$, thus $T(n) = \Theta (n^{3})$.\\
  b. $\text{log}_{b}a = \text{log}_{2}8 = 3 = c$, thus $T(n) = \Theta (n^{3}\text{log}~n)$.\\
  c. $\text{log}_{b}a = \text{log}_{2}3 > 1$, thus $T(n) = \Theta (n^{\text{log}_{2}3})$.\\
  d. $\text{log}_{b}a = \text{log}_{4}1 = 0 = c$, thus $T(n) = \Theta (\text{log}~n)$.\\
  e. $\text{log}_{b}a = \text{log}_{3}3 = 1 < 2$, thus $T(n) = \Theta (n^{2})$.
\end{solution}
%%%%%%%%%

%%%%%%%%%%
\begin{problem}[CS: 4.4.6]
  Extend the proof of the preliminary version of the master theorem to the 
  case $T(1) = d$.
\end{problem}

\begin{solution}
  In Case 1, $T(n) = a^{\text{log}_{b}n}d + n^{c}\sum_{i = 0}^{\text{log}_{b}n - 1} (\frac{a}{b^{c}})^{i} 
  = d\cdot n^{\text{log}_{b}a} + \Theta (n^{c}) = \Theta (n^{c})$;\\
  In Case 2, $T(n) = a^{\text{log}_{b}n}d + n^{c}\sum_{i = 0}^{\text{log}_{b}n - 1} (\frac{a}{b^{c}})^{i} 
  = d\cdot n^{c} + \Theta (n^{c} \text{log}~n) = \Theta (n^{c} \text{log}~n)$;\\
  In Case 3, $T(n) = a^{\text{log}_{b}n}d + n^{c}\sum_{i = 0}^{\text{log}_{b}n - 1} (\frac{a}{b^{c}})^{i} 
  = d\cdot n^{\text{log}_{b}a} + \Theta (n^{\text{log}_{b}a}) = \Theta (n^{\text{log}_{b}a})$.\\
  Proof completed.
\end{solution}
%%%%%%%%%

%%%%%%%%%
\begin{problem}[CS: 4.5.8]
  Show by induction that
  \[ T(n) = 
  \begin{cases}
    8T(n/2) + n\text{ log }n & \text{if } n > 1, \\
    d & \text{if } n = 1,
  \end{cases} \]
  has $T(n) = O(n^{3})$ for any solution $T(n)$.
\end{problem}

\begin{solution}
  Suppose $T(n) \leq c_{1}n^{3} - c_{2}n^{2}$ for any $n \geq 1$.\\
  For $n = 1$, $c_{1}n^{3} - c_{2}n^{2} = c_{1} - c_{2}$, if $c_{1} - c_{2} \geq d$, 
  then $T(1) \leq c_{1}1^{3} - c_{2}1^{2}$;\\
  Suppose for any $1 \leq k < n$, $T(k) \leq c_{1}k^{3} - c_{2}l^{2}$, then 
  for $n$, we have
  \begin{align}
    T(n) & = 8T(n/2) + n \text{ log } n \notag \\
         & \leq 8(c_{1}(\frac{n}{2})^{3} - c_{2}(\frac{n}{2})^{2}) + n \text{ log } n \notag \\
         & = c_{1}n^{3} - (2c_{2} - 1)n(n - \text{log }n) \notag
  \end{align}
  for $n > 1$, $n > \text{log }n$, then if $c_{2} \geq \frac{1}{2},~c_{1}\geq \frac{1}{2} + d$, 
  we have
  \begin{align}
    T(n) & \leq c_{1}n^{3} - (2c_{2} - 1)n(n - \text{log }n) \notag \\
         & \leq c_{1}n^{3} \notag \\
         & = O(n^{3}) \notag
  \end{align}
  Thus we can conclude that for all $n \geq 1$, $T(n) = O(n^{3})$.
\end{solution}
%%%%%%%%%

%%%%%%%%%
\begin{problem}[CS: 4.5.9]
  Is the big $O$ upper bound in Problem 8 actually a big $\Theta$ bound?
\end{problem}

\begin{solution}
  Prove that $T(n) \geq cn^{3}$ for all $n \geq 1$.\\
  For $n = 1$, $cn^{3} = c$, if $c \leq d$, then $T(1) \geq c\cdot 1^{3}$;\\
  Suppose for any $1 \leq k < n$, $T(k) \geq ck^{3}$, then for $n$, we have
  \begin{align}
    T(n) & = 8T(n/2) + n \text{ log } n \notag \\
         & \geq 8c(\frac{n}{2})^{3} + n \text{ log } n \notag \\
         & = cn^{3} + n \text{log }n \notag
  \end{align}
  for $n > 1$, $n^{2} > \text{log }n$, then for $0 < c \leq d$, we have
  \begin{align}
    T(n) & \geq cn^{3} + n \text{log }n \notag \\
         & \geq cn^{3} \notag \\
         & = \Omega (n^{3}) \notag
  \end{align}
  Thus we can conclude that for all $n \geq 1$, $T(n) = \Omega (n^{3})$.\\
  plus the conclusion in Problem 8, we can find that $T(n) = \Theta (n^{3})$ 
  for all $n \geq 1$.
\end{solution}
%%%%%%%%%

%%%%%%%%%
\begin{problem}[CS: 4.5.10]
  Give the best big $O$ upper bound you can for the solution to the recurrence
  \[ T(n) = 2T(\frac{n}{3} - 3) + n. \]
  Then prove by induction that your upper bound is correct.
\end{problem}

\begin{solution}
  The upper bound should be $O(n)$.\\
  Prove that for all $n \geq 1$, $T(n) \leq cn$.\\
  Assume that $\text{max} \{ T(1),T(2),\cdots,T(11) \} = d$,\\
  For $n = 1,2,\cdots,11$, if $c \geq d$, then $T(n) \leq cn$;\\
  Suppose for $11 \leq k < n$, $T(k) \leq ck$, then for $n$, we have
  \begin{align}
    T(n) & = 2T(\frac{n}{3} - 3) + n \notag \\
         & \leq 2c[\frac{n}{3} - 3] + n \notag \\
         & = \frac{5}{3}cn - 6c \notag
  \end{align}
  for $c \geq 1$, we have
  \begin{align}
    T(n) & \leq \frac{5}{3}cn - 6c \notag \\
         & \leq \frac{5}{3}cn \notag \\
         & = O(n) \notag
  \end{align}
  Thus for all $n \geq 1$, $T(n) = O(n)$.
\end{solution}
%%%%%%%%%%%%%%%%%%%%

%%%%%%%%%%%%%%%%%%%%
\end{document}